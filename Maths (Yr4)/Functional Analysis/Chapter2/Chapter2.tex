\documentclass[a4paper, openany]{memoir}

\usepackage[utf8]{inputenc}
\usepackage[T1]{fontenc} 
\usepackage[english]{babel}

\usepackage{fancyhdr}
\usepackage{float}
\usepackage{bm}

\usepackage{amsmath}
\usepackage{amsthm}
\usepackage{amssymb}
\usepackage{enumitem}
\usepackage{multicol}
\usepackage[bookmarksopen=true,bookmarksopenlevel=2]{hyperref}
\usepackage{tikz}
\usepackage{indentfirst}

\pagestyle{fancy}
\fancyhf{}
\fancyhead[LE]{\leftmark}
\fancyhead[RO]{\rightmark}
\fancyhead[RE, LO]{Functional Analysis}
\fancyfoot[LE, RO]{\thepage}
\fancyfoot[RE, LO]{Pete Gautam}

\renewcommand{\headrulewidth}{1.5pt}

\theoremstyle{definition}
\newtheorem{definition}{Definition}[section]

\theoremstyle{plain}
\newtheorem{theorem}[definition]{Theorem}
\newtheorem{lemma}[definition]{Lemma}
\newtheorem{proposition}[definition]{Proposition}
\newtheorem{corollary}[definition]{Corollary}
\newtheorem{example}[definition]{Example}

\chapterstyle{thatcher}
\setcounter{chapter}{1}

\begin{document}
    \chapter{Measure Theory}
    \section{Rings and Algebras}
    In this section, we will define measures and see why they are necessary. In relation to functional analysis, measure theory is needed to formally define function spaces that complement the sequence spaces $\ell^p$. We will be able to extend $C[0, 1]$ to a bigger set of functions by defining integration for more general spaces.

    Intuitively, a measure on $\mathbb{R}$ is a function $\mu \colon \mathbb{P}(\mathbb{R}) \to [0, \infty]$ such that
    \begin{itemize}
        \item for disjoint collection of countable sets $(A_i)_{i \in I}$,
        \[\mu \left(\bigcup_{i \in I} A_i\right) = \sum_{i \in I} \mu(A_i);\]
        \item for any $x \in \mathbb{R}$ and $A \subseteq \mathbb{R}$, $\mu(x + A) = \mu(A)$ (translation-invariant); and
        \item the measure of the unit interval $\mu([0, 1]) = 1$.
    \end{itemize}
    It turns out that it is not possible to find such a function. We will prove this by showing that satisfying the first 2 axioms implies that the third axiom cannot be satisfied.

    So, assume that we have a measure function $\mu$ on $\mathbb{R}$ satisfying the first two axioms. Then, define the equivalence relation $\sim$ on $[0, 1]$ by 
    \[x \sim y \iff x - y \in \mathbb{Q}.\]
    From each equivalence class, choose a specific element $x$, and let $N$ be the set containing all such elements. Now, for $r \in \mathbb{Q} \cap [0, 1)$, define
    \[N_r = \{x + r \mid x \in N, x \leq 1 - r\} \cup \{x + r - 1 \mid x \in N, x > 1 - r\}.\]
    We know that $N_r \cap N_q \neq \varnothing$ if and only if $r = q$, with
    \[\bigcup_{r \in \mathbb{Q} \cap [0, 1)} N_r = [0, 1).\]
    Since the measure function $\mu$ is translation-invariant, we know that $\mu(N_r) = N$ for all $r \in \mathbb{Q} \cap [0, 1]$. Moreover, since $\mathbb{Q} \cap [0, 1]$ is countable with $N_r$ disjoint for all $r \in \mathbb{Q} \cap [0, 1]$, we find that
    \[\mu([0, 1)) = \mu \left(\bigcup_{r \in \mathbb{Q} \cap [0, 1)}\right) = \sum_{r \in \mathbb{Q} \cap [0, 1)} \mu(N_r) = \sum_{r \in \mathbb{Q} \cap [0, 1)} \mu(N).\]
    The value $\mu(N)$ is a constant, so either $\mu(N) = 0$, in which case $\mu([0, 1)) = 0$, or $\mu(N) > 0$, in which case $\mu([0, 1)) = \infty$. We would like to give the interval $[0, 1)$ a non-zero finite value (in particular 1), and to do so, we cannot allow every subset of $\mathbb{R}$ to be measurable.

    We will now define the theory of rings and algebras, and later use this to define measurable functions, including the one we want in $\mathbb{R}^n$.
    \begin{definition}
        Let $X$ be a set and let $\mathcal{A} \subseteq \mathbb{P}(X)$ be a non-empty collection of subsets of $X$. 
        \begin{itemize}
            \item We say that $(X, \mathcal{A})$ is a \emph{ring} if for all $A, B \in \mathcal{A}$, $A \cup B \in \mathcal{A}$ and $A \setminus B \in \mathcal{A}$.
            \item We say that $(X, \mathcal{A})$ is an \emph{algebra} if for all $A, B \in \mathcal{A}$, $A \cup B \in \mathcal{A}$ and $A^c \in \mathcal{A}$.
            \item We say that $(X, \mathcal{A})$ is a \emph{$\sigma$-algebra} if for all $A \in \mathcal{A}$, $A^c \in \mathcal{A}$, and for a collection $(A_i)_{i=1}^\infty$ of sets in $\mathcal{A}$, the union
            \[\bigcup_{i=1}^\infty A_i \in \mathcal{A}.\]
        \end{itemize}
    \end{definition}
    We will only be focusing on $\sigma$-algebras, but many of the results will hold for rings or algebras. A $\sigma$-algebra is an algebra, and an algebra is a ring. An example of a $\sigma$-algebra is $\mathcal{P}(X)$, for some set $X$. For any set $A \subseteq X$, the smallest $\sigma$-algebra containing $A$ is denoted $\sigma(A)$. It is given by the intersection of $\sigma$-algebras containing $A$- this is always a $\sigma$-algebra. For a topological space $X$, the set $\mathcal{B}(X)$ denotes the Borel sets, which is the $\sigma$-algebra generated by open sets in $X$.

    We want to define measurable functions on $\sigma$-algebras. This can be defined on a ring.
    \begin{definition}
        Let $(X, \mathcal{A})$ be a ring, and let $\mu \colon \mathcal{A} \to [0, \infty]$ be a function.
        \begin{itemize}
            \item We say that $\mu$ is \emph{additive} if for $A, B \in \mathcal{A}$ disjoint, $\mu(A \cup B) = \mu(A) + \mu(B)$.
            \item We say that $\mu$ is \emph{$\sigma$-subadditive} if for a collection of disjoint sets $(A_i)_{i=1}^\infty$,
            \[\mu \left(\bigcup_{i=1}^\infty A_i\right) \leq \sum_{i=1}^\infty \mu(A_i)\]
            if the set $\bigcup_{i=1}^\infty A_i \in \mathcal{A}$.
            \item We say that $\mu$ is \emph{$\sigma$-additive} if for a collection of disjoint sets $(A_i)_{i=1}^\infty$,
            \[\mu \left(\bigcup_{i=1}^\infty A_i\right) = \sum_{i=1}^\infty \mu(A_i)\]
            if the set $\bigcup_{i=1}^\infty A_i \in \mathcal{A}$.
        \end{itemize}
    \end{definition}
    An example of a measurable function on the natural numbers $\mathbb{Z}_{\geq 1}$ is the counting measure i.e. $m(A) = |A|$. In this case, it is possible to define the measure on the entire set. We will later see that integration defined on this set will precisely give us the $\ell^p$ sequence spaces.

    Now, let $\mathcal{E}(\mathbb{R})$ be the set of finite union of intervals in $\mathbb{R}$. This is a ring- the union of two unions is still a union, and the set difference of two intervals is a union of intervals. However, it is not an algebra since it does not contain countable union of intervals. We can define the following measure $\mu$ on $\mathcal{E}(\mathbb{R})$. First, for an interval $I_k$, we define
    \[\mu(I_k) = \begin{cases}
        \infty & I_k \textrm{ not bounded} \\
        \sup I_k - \inf I_k & \textrm{otherwise}.
    \end{cases}\]
    Now, for a union of intervals
    \[I = \bigcup_{k=1}^n I_k,\]
    we can make it a disjoint union of intervals
    \[I = \bigcup_{k=1}^n J_k\]
    (a possible choice is $J_1 = I_1$, $J_2 = I_2 \setminus J_1$, $J_3 = I_3 \setminus (J_1 \cup J_2)$ and so on). Then, we define
    \[\mu(I) = \sum_{k=1}^n \mu(J_k).\]
    By definition, $\mu$ is additive. It is also $\sigma$-additive. We will later extend $\mu$ to the $\sigma$-algebra generated by $\mathcal{E}(\mathbb{R})$.


\end{document}
