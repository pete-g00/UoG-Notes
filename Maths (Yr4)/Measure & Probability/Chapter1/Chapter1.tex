\documentclass[a4paper, openany]{memoir}

\usepackage[utf8]{inputenc}
\usepackage[T1]{fontenc} 
\usepackage[english]{babel}

\usepackage{fancyhdr}
\usepackage{float}
\usepackage{bm}

\usepackage{amsmath}
\usepackage{amsthm}
\usepackage{amssymb}
\usepackage{enumitem}
\usepackage{multicol}
\usepackage[bookmarksopen=true,bookmarksopenlevel=2]{hyperref}
\usepackage{tikz}
\usepackage{indentfirst}

\pagestyle{fancy}
\fancyhf{}
\fancyhead[LE]{\leftmark}
\fancyhead[RO]{\rightmark}
\fancyhead[RE, LO]{Measure \& Probability}
\fancyfoot[LE, RO]{\thepage}
\fancyfoot[RE, LO]{Pete Gautam}

\renewcommand{\headrulewidth}{1.5pt}

\theoremstyle{definition}
\newtheorem{definition}{Definition}[section]
\newtheorem{example}[definition]{Example}

\theoremstyle{plain}
\newtheorem{theorem}[definition]{Theorem}
\newtheorem{lemma}[definition]{Lemma}
\newtheorem{proposition}[definition]{Proposition}
\newtheorem{corollary}[definition]{Corollary}

\chapterstyle{thatcher}

\begin{document}
    \chapter{Rings and Algebras}
    \setcounter{section}{-1}
    \section{Recap of Set Theory}
    \begin{definition}
        Let $X$ be a set. Then, the \emph{powerset of $X$} is the set of subsets of $X$, and is denoted by $\mathcal{P}(X)$.
    \end{definition}

    \begin{definition}
        Let $X$ be a set, and let $(X_i)_{i \in I}$ be a collection of subsets of $X$, for some indexing set $I$. We define the \emph{union} to be:
        \[\bigcup_{i \in I} X_i = \{x \in X \mid \exists i \in I \text{ s.t. } x \in X_i\}.\]
        Similarly, we define the \emph{intersection} to be:
        \[\bigcap_{n=1}^\infty A_n = \{x \in X \mid \forall i \in I \text{ s.t. } x \in X_i\}.\]
    \end{definition}

    \begin{definition}
        Let $X$ be a set, and let $A \subseteq X$. We define the \emph{complement of $A$} to be:
        \[A^c = X \setminus A = \{x \in X \mid x \not\in A\}.\]
    \end{definition}

    \begin{proposition}[De Morgan Law]
        Let $A$ and $B$ be sets. Then,
        \[(A \cup B)^c = A^c \cap B^c, \qquad (A \cap B)^c = A^c \cup B^c.\]
        In general, for a collection of sets $(A_i)_{i \in I}$, where $I$ is an index set,
        \[\left(\bigcup_{i \in I} A_i\right)^c = \bigcap_{i \in I} A_i^c, \qquad \left(\bigcap_{i \in I} A_i\right)^c = \bigcup_{i \in I} A_i^c.\]
    \end{proposition}

    \begin{definition}
        Let $S$ be a set. We say that \emph{$S$ is countable} if either $S$ is empty, or there exists a surjective function $f: \mathbb{Z}_{\geq 1} \to S$. If so, we can denote
        \[S = \{f(1), f(2), f(3), \dots\}.\]
    \end{definition}

    \begin{proposition}
        Let $S$ be a countable set, and let $T \subseteq S$. Then, $T$ is countable.
    \end{proposition}

    \begin{proposition}
        Let $S$ be a countably infinite set. Then, there exists a bijective function $f: \mathbb{Z}_{\geq 1} \to S$.
    \end{proposition}

    \begin{proposition}
        Let $S$ and $T$ be countable sets. Then, their union $S \cup T$ is countable.
    \end{proposition}

    \begin{proposition}
        Let $(S_n)_{n=1}^\infty$ be a sequence of countable sets. Then, their union 
        \[\bigcup_{n=1}^\infty S_n\]
        is countable.
    \end{proposition}

    \begin{proposition}
        Let $S$ and $T$ be countable sets. Then, the product $S \times T$ are countable.
    \end{proposition}

    \begin{corollary}
        The set $\mathbb{Q}$ is countable.
    \end{corollary}

    \begin{proposition}
        The set $[0, 1]$ is not countable.
    \end{proposition}

    \begin{definition}
        Let $A$ and $B$ be sets. We say that $|A| = |B|$ if there exists a bijection $f: A \to B$. If there exists an injective function $f: A \to B$, then we say that $|A| \leq |B|$.
    \end{definition}
    \newpage

    \section{Rings and Algebras}
    \begin{definition}
        Let $X$ be a set. We say that $\mathcal{R} \subseteq \mathcal{P}(X)$ is a \emph{ring} (of subsets of $X$) if:
        \begin{itemize}
            \item $\varnothing \in \mathcal{R}$;
            \item for all $A, B \in \mathcal{R}$, the difference $A \setminus B \in \mathcal{R}$;
            \item for all $A, B \in \mathcal{R}$, the union $A \cup B \in \mathcal{R}$.
        \end{itemize}
    \end{definition}

    \begin{proposition}
        Let $X$ be a set, and let $\mathcal{R} \subseteq \mathcal{P}(X)$ be a ring. Then, for $A, B \in \mathcal{R}$, the intersection $A \cap B \in \mathcal{R}$.
    \end{proposition}
    \begin{proof}
        We have
        \[A \cap B = (A \cup B) \setminus (A \setminus B \cup B \setminus A).\]
        So, $A \cap B \in \mathcal{R}$.
    \end{proof}
    
    \begin{definition}
        Let $X$ be a set. We say that $\mathcal{A} \subseteq \mathcal{P}(X)$ is an \emph{algebra} (of subsets of $X$) if $\mathcal{A}$ is a ring with $X \in \mathcal{A}$.
    \end{definition}

    \begin{proposition}
        Let $X$ be a set, and $\mathcal{A} \subseteq \mathcal{P}(X)$. Then, $\mathcal{A}$ is an algebra if and only if:
        \begin{itemize}
            \item $\varnothing \in \mathcal{A}$;
            \item for all $A \in \mathcal{A}$, the complement $A^c \in \mathcal{A}$; and
            \item for all $A, B \in \mathcal{A}$, the union $A \cup B \in \mathcal{A}$.
        \end{itemize}
    \end{proposition}
    \begin{proof}
        \hspace{0pt}
        \begin{itemize}
            \item[$\implies$] Since $\mathcal{A}$ is an algebra, we know that $\varnothing \in \mathcal{A}$, and for all $A, B \in \mathcal{A}$, $A \cup B \in \mathcal{A}$. Now, let $A \in \mathcal{A}$. Since $X \in \mathcal{A}$, we find that $A^c = X \setminus A \in \mathcal{A}$.
            \item[$\impliedby$] We know that $\varnothing \in \mathcal{A}$, and for all $A, B \in \mathcal{A}$, $A \cup B \in \mathcal{A}$. Now, let $A \in \mathcal{A}$. We know that $A \setminus B = A \cap B^c$.
        \end{itemize}
        So, the result holds.
    \end{proof}

    \begin{definition}
        Let $X$ be a set. We say that $\mathcal{A} \subseteq \mathcal{P}(X)$ is a \emph{$\sigma$-algebra} (of subsets of $X$) if $\mathcal{A}$ is an algebra such that for all $(A_n)_{n=1}^\infty$ in $\mathcal{A}$, the union
        \[\bigcup_{n=1}^\infty A_n \in \mathcal{A}.\]
    \end{definition}

    \begin{proposition}
        Let $X$ be a set, and $\mathcal{A} \subseteq \mathcal{P}(X)$. Then, $\mathcal{A}$ is a $\sigma$-algebra if and only if:
        \begin{itemize}
            \item $\varnothing \in \mathcal{A}$;
            \item for all $A \in \mathcal{A}$, the complement $A^c \in \mathcal{A}$; and
            \item for a sequence $(A_n)_{n=1}^\infty$ in $\mathcal{A}$, the union
            \[\bigcup_{n=1}^\infty A_n \in \mathcal{A}.\]
        \end{itemize}
    \end{proposition}
    \begin{proof}
        \hspace*{0pt}
        \begin{itemize}
            \item[$\implies$] 
            \item[$\impliedby$] 
        \end{itemize}
    \end{proof}

    \begin{proposition}
        Let $X$ be a set, and $\mathcal{A} \subseteq \mathcal{P}(X)$ be a $\sigma$-algebra. Then, for a sequence $(A_n)_{n=1}^\infty$ in $\mathcal{A}$, the intersection
        \[\bigcap_{n=1}^\infty A_n \in \mathcal{A}.\]
    \end{proposition}
    \begin{proof}
        Define the sequence $(B_n)_{n=1}^\infty$ by $B_n = A_n^c$. Since $\mathcal{A}$ is an algebra, $(B_n)$ is in $\mathcal{A}$. Moreover, 
        \[\bigcup_{n=1}^\infty B_n \in \mathcal{A}.\]
        Hence,
        \[\bigcap_{n=1}^\infty A_n = \left(\bigcup_{n=1}^\infty B_n\right)^c \in \mathcal{A}.\]
    \end{proof}
    
    \newpage

    \section{Borel Sets}
    \begin{definition}
        We define $\mathcal{E}(\mathbb{R})$ to be the set containing all finite unions of intervals in $\mathbb{R}$.
    \end{definition}

    \begin{proposition}
        The set $\mathcal{E}(\mathbb{R})$ is a ring.
    \end{proposition}
    \begin{proof}
        
    \end{proof}

    \begin{definition}
        Let $n \in \mathbb{Z}_{\geq 1}$. We define $\mathcal{E}(\mathbb{R}^n)$ to be the set containing all finite union of intervals in $\mathbb{R}^n$, where an interval in $\mathbb{R}^n$ is a product of $n$ intervals in $\mathbb{R}$.
    \end{definition}

    \begin{proposition}
        The set $\mathcal{E}(\mathbb{R}^n)$ is a ring.
    \end{proposition}

    \begin{definition}
        We define the \emph{Borel set} $\mathcal{B}(\mathbb{R})$ to be the $\sigma$-algebra generated by $\mathcal{E}(\mathbb{R})$.
    \end{definition}

    \begin{proposition}
        Let $A \in \mathcal{B}(\mathbb{R})$ and $x \in \mathbb{R}$. Then,
        \[x + A = \{x + a \mid a \in A\} \in \mathcal{B}(\mathbb{R}).\]
    \end{proposition}
    \begin{proof}
        Let $x \in \mathbb{R}$. Define the set
        \[\mathcal{A} = \{A \in \mathcal{B}(\mathbb{R}) \mid x + A \in \mathcal{B}(\mathbb{R})\}.\]
        We show that $\mathcal{A}$ is a $\sigma$-algebra.
        \begin{itemize}
            \item We have $x + \varnothing = \varnothing$, so $\varnothing \in \mathcal{A}$;
            
            \item Let $A \in \mathcal{A}$. For $y \in \mathbb{R}$,
            \begin{align*}
                y \in x + A^c &\iff y - x \in A^c \\
                &\iff y - x \not\in A \\
                &\iff y \not\in x + A \\
                &\iff y \in (x + A)^c.
            \end{align*}
            So, $(x + A)^c = x + A^c$. Since $x + A \in \mathcal{B}(\mathbb{R})$, we have 
            \[x + A^c = (x + A)^c \in \mathcal{B}(\mathbb{R}).\]
            Hence, $A^c \in \mathcal{A}$.
            
            \item Let $(A_n)_{n=1}^\infty$ be a sequence of disjoint sets in $\mathcal{A}$. For $y \in \mathbb{R}$,
            \begin{align*}
                y \in \bigcup_{n=1}^\infty (x + A_n) &\iff \exists n \in \mathbb{Z}_{\geq 1} \textrm{ s.t. } y \in x + A_n \\
                &\iff y - x \in A_n \\
                &\iff y - x \in \bigcup_{n=1}^\infty A_n \\
                &\iff y \in x + \bigcup_{n=1}^\infty A_n.
            \end{align*}
            Since $x + A_n \in \mathcal{B}(\mathbb{R})$ for all $n \in \mathbb{Z}_{\geq 1}$,
            \[x + \bigcup_{n=1}^\infty A_n = \bigcup_{n=1}^\infty (x + A_n) \in \mathcal{B}(\mathbb{R}).\]
            This implies that
            \[\bigcup_{n=1}^\infty A_n \in \mathcal{A}.\]
        \end{itemize}
        Hence, $\mathcal{A}$ is a $\sigma$-algebra. Moreover, for all interval $I$, $x + I \in \mathcal{B}(\mathbb{R})$. Hence, $\mathcal{A}$ contains all the intervals. Since $\mathcal{A}$ is a $\sigma$-algebra, we find that $\mathcal{A} = \mathcal{B}(\mathbb{R})$. This implies that for all $A \in \mathcal{B}(\mathbb{R})$, $x + A \in \mathcal{B}(\mathbb{R})$.
    \end{proof}
    
    \newpage

    \section{Measure on Algebra}
    \begin{definition}
        Let $X$ be a set and $\mathcal{R}$ be a ring of subsets of $X$. We say that $\mu \colon \mathcal{R} \to [0, \infty]$ is an \emph{additive set function} if:
        \begin{itemize}
            \item $\mu(\varnothing) = 0$ and
            \item for all $A, B \in \mathcal{R}$ with $A \cap B = \varnothing$, $\mu(A \cup B) = \mu(A) + \mu(B)$.
        \end{itemize}
    \end{definition}

    \begin{proposition}
        Let $X$ be a set and $\mathcal{R}$ be a ring of subsets of $X$, and $\mu \colon \mathcal{R} \to [0, \infty]$ be an additive set function. Then, for $A, B \in \mathcal{R}$,
        \begin{enumerate}
            \item if $A \subseteq B$ then $\mu(A) \leq \mu(B)$;
            \item $\mu(A \cup B) \leq \mu(A) + \mu(B)$.
        \end{enumerate}
    \end{proposition}
    \begin{proof}
        \hspace{0pt}
        \begin{enumerate}
            \item We have
            \[\mu(B) = \mu(A) + \mu(B \setminus A) \leq \mu(A).\]
            \item We find that
            \[\mu(A \cup B) = \mu(A) + \mu(B \setminus A) \leq \mu(A) + \mu(B)\]
            since $B \setminus A \subseteq B$.
        \end{enumerate}
    \end{proof}

    \begin{definition}
        Let $X$ be a set and $\mathcal{R}$ be a ring of subsets of $X$. We say that $\mu \colon \mathcal{R} \to [0, \infty]$ is a \emph{measure} if:
        \begin{itemize}
            \item $\mu(\varnothing) = 0$ and
            \item for a sequence $(A_n)_{n=1}^\infty$ in $\mathcal{R}$ of pairwise disjoint sets, if $\bigcup_{n=1}^\infty A_n \in \mathcal{R}$, then
            \[\mu \left(\bigcup_{n=1}^\infty A_n\right) = \sum_{n=1}^\infty \mu(A_n).\]
        \end{itemize}
    \end{definition}
    
    \begin{proposition}
        Let $X$ be a set, $\mathcal{R}$ be a ring of subsets of $X$ and let $\mu \colon \mathcal{R} \to [0, \infty]$ be a measure. Then, for all sequences $(A_n)_{n=1}^\infty$ in $\mathcal{R}$,
        \[\mu \left(\bigcup_{n=1}^\infty A_n\right) \leq \sum_{n=1}^\infty \mu(A_n).\]
    \end{proposition}
    \begin{proof}
        Define the sequence $(B_n)_{n=1}^\infty$ inductively by $B_1 = A_1$ and 
        \[B_n = A_n \setminus \bigcup_{k=1}^{n-1} B_k\]
        for $n \geq 2$. Then, $(B_n)$ is a sequence of disjoint sets in $\mathcal{R}$ such that 
        \[\bigcup_{n=1}^\infty A_n = \bigcup_{n=1}^\infty B_n.\]
        Hence,
        \[\mu \left(\bigcup_{n=1}^\infty A_n\right) = \mu \left(\bigcup_{n=1}^\infty B_n\right) = \sum_{n=1}^\infty \mu(B_n) \leq \sum_{n=1}^\infty \mu(A_n)\]
        since $B_n \subseteq A_n$ for all $n \in \mathbb{Z}_{\geq 1}$.
    \end{proof}

    \begin{definition}
        Let $X$ be a set, $\mathcal{R}$ a ring of subsets of $X$, and $\mu \colon \mathcal{R} \to [0, \infty]$ be an additive set function. We say that $\mu$ is \emph{$\sigma$-finite} if there exists a sequence $(A_n)_{n=1}^\infty$ in $\mathcal{R}$ such that $\mu(A_n) < \infty$ for all $n \in \mathbb{Z}_{\geq 1}$, and
        \[X = \bigcup_{n=1}^\infty A_n.\]
        If we have $X \in \mathcal{R}$ with $\mu(X) < \infty$, then $\mu$ is \emph{finite}.
    \end{definition}

    \begin{proposition}
        Let $X$ be a set, $\mathcal{R}$ a ring of subsets of $X$, and $\mu \colon \mathcal{R} \to [0, \infty)$ be an additive set function. Then, the following are equivalent:
        \begin{enumerate}
            \item $\mu$ is countably additive (i.e. a measure);
            \item If $(A_n)_{n=1}^\infty$ is a sequence in $\mathcal{R}$ with $A_n \subseteq A_{n+1}$ for all $n \in \mathbb{Z}_{\geq 1}$ with
            \[A = \bigcup_{n=1}^\infty A_n \in \mathcal{R},\]
            then
            \[\mu(A) = \lim_{n \to \infty} \mu(A_n).\]
            \item If $(A_n)_{n=1}^\infty$ is a sequence in $\mathcal{R}$ with $A_n \supseteq A_{n+1}$ for all $n \in \mathbb{Z}_{\geq 1}$ with
            \[\bigcap_{n=1}^\infty A_n = A,\]
            then
            \[\mu(A) = \lim_{n \to \infty} \mu(A_n).\]
            \item If $(A_n)_{n=1}^\infty$ is a sequence in $\mathcal{R}$ with $A_n \supseteq A_{n+1}$ for all $n \in \mathbb{Z}_{\geq 1}$ with
            \[\bigcap_{n=1}^\infty A_n = \varnothing,\]
            then
            \[\lim_{n \to \infty} \mu(A_n) = 0 = \mu(\varnothing).\]
        \end{enumerate}
    \end{proposition}
    \begin{proof}
        \hspace{0pt}
        \begin{itemize}
            \item[$(1) \implies (2)$] Define the sequence $(B_n)_{n=1}^\infty$ in $\mathcal{R}$ inductively by $B_1 = A_1$ and $B_n = A_n \setminus B_{n-1}$ for $n \geq 2$. Then, $(B_n)$ is a sequence of disjoint sets with
            \[\bigcup_{n=1}^\infty A_n = \bigcup_{n=1}^\infty B_n \qquad \textrm{and} \qquad \bigcup_{n=1}^N A_n = A_N\]
            for all $N \geq 1$. Hence,
            \begin{align*}
                \mu(A) &= \mu \left(\bigcup_{n=1}^\infty A_n\right) \\
                &= \mu \left(\bigcup_{n=1}^\infty B_n\right) \\
                &= \sum_{n=1}^\infty \mu(B_n) \\
                &= \lim_{N \to \infty} \sum_{n=1}^N \mu(B_n) \\
                &= \lim_{N \to \infty} \mu \left(\bigcup_{n=1}^N B_n\right) \\
                &= \lim_{N \to \infty} \mu(A_N).
            \end{align*}

            \item[$(2) \implies (3)$] Define the sequence $(B_n)_{n=1}^\infty$ in $\mathcal{R}$ by $B_1 = A_1^c$. Then, $B_n \subseteq B_{n+1}$ for all $n \in \mathbb{Z}_{\geq 1}$ with
            \[\bigcup_{n=1}^\infty B_n = \bigcup_{n=1}^\infty A_n^c = \left(\bigcap_{n=1}^\infty A_n\right)^c = A^c.\]
            Hence,
            \[\mu(A^c) = \lim_{n \to \infty} \mu(B_n).\]
            This implies that
            \begin{align*}
                \mu(A) &= \mu(X) - \mu(A^c) \\
                &= \mu(X) - \lim_{n \to \infty} \mu(A_n^c) \\
                &= \mu(X) - \lim_{n \to \infty} \mu(X) - \mu(A_n) \\
                &= \lim_{n \to \infty} \mu(A_n).
            \end{align*}
            
            \item[$(3) \implies (4)$] This follows if we set $A = \varnothing$.
            
            \item[$(4) \implies (1)$] Let $(A_n)_{n=1}^\infty$ be a sequence in $\mathcal{R}$ of disjoint sets. Define the sequence $(B_n)_{n=1}^\infty$ in $\mathcal{R}$ by
            \[B_n = \bigcup_{k=n}^\infty A_k.\]
            By definition, $B_{n+1} \supset B_{n+1}$ for all $n \in \mathbb{Z}_{\geq 1}$. Moreover, we claim that
            \[\bigcap_{n=1}^\infty B_n = \varnothing.\]
            So, assume for a contradiction, that
            \[x \in \bigcap_{n=1}^\infty B_n.\]
            In that case, $x \in B_1$. In particular, there exists an $k \in \mathbb{Z}_{\geq 1}$ such that $x \in A_k$. Since $(A_n)$ is a sequence of disjoint sets, we know that $x \not\in A_n$ for $n \geq k+1$. Hence, $x \not\in B_{k+1}$. This is a contradiction. So, 
            \[\bigcap_{n=1}^\infty B_n = \varnothing.\]
            This implies that
            \[\lim_{n \to \infty} \mu(B_n) = 0.\]
            Hence,
            \begin{align*}
                \sum_{n=1}^\infty \mu(A_n) &= \lim_{N \to \infty} \sum_{n=1}^N \mu(A_n) \\
                &= \lim_{N \to \infty} \mu \left(\bigcup_{n=1}^N A_n\right) \\
                &= \lim_{N \to \infty} \mu \left(\bigcup_{n=1}^\infty A_n \setminus B_N\right) \\
                &= \lim_{N \to \infty} \mu \left(\bigcup_{n=1}^\infty A_n\right) - \mu(B_N) \\
                &= \mu \left(\bigcup_{n=1}^\infty A_n\right).
            \end{align*}
        \end{itemize}
    \end{proof}

    \begin{definition}
        We define the \emph{Lebesgue measure} $\lambda \colon \mathcal{E}(\mathbb{R}) \to [0, \infty]$ as the extension of $\lambda(I) = \sup I - \inf I$, for some interval $I$. In particular, for disjoint intervals $(I_k)_{k=1}^n$, we define
        \[\lambda \left(\bigcup_{k=1}^n I_k\right) = \sum_{k=1}^n \lambda (I_k).\]
    \end{definition}

    \begin{lemma}
        The Lebesgue measure $\lambda \colon \mathcal{E}(\mathbb{R}) \to [0, \infty]$ is well-defined.
    \end{lemma}
    \begin{proof}
        % TODO
        % Let $(I_k)_{k=1}^n$ and $(J_l)_{l=1}^m$ be two sequences of disjoint intervals such that 
        % \[\bigcup_{k=1}^n I_k = \bigcup_{l=1}^m J_l.\]
        % Now, for all $1 \leq k \leq n$,
        % \[I_k = I_k \cap \left(\bigcup_{l=1}^n I_l\right) = I_k \cap \left(\bigcup_{l=1}^m J_l\right) = \bigcup_{l=1}^m (I_k \cap J_l).\]
        % So,
        % \[\bigcup_{k=1}^n I_k = \bigcup_{k=1}^n \bigcup_{l=1}^m (I_k \cap J_l).\]
        % Similarly,
        % \[\bigcup_{l=1}^n J_l = \bigcup_{l=1}^m \bigcup_{k=1}^n (I_k \cap J_l).\]
    \end{proof}

    \begin{lemma}
        Let $A \in \mathcal{E}(\mathbb{R})$ with $\lambda(A) > 0$. Then, for all $\delta \in (0, 1)$, there exists a closed $A' \in \mathcal{E}(\mathbb{R})$ such that $A' \subseteq A$ and $\lambda(A') = (1 - \delta ) \lambda(A)$. In particular, for every $\varepsilon > 0$, there exists a closed $A' \in \mathcal{E}(\mathbb{R})$ such that $\lambda(A \setminus A') < \varepsilon$.
    \end{lemma}
    \begin{proof}
        
    \end{proof}

    \begin{theorem}
        The Lebesgue measure $\lambda \colon \mathcal{E}(\mathbb{R}) \to [0, \infty]$ is a measure.
    \end{theorem}
    \begin{proof}
        
    \end{proof}
    \newpage

    \section{Outer Measure}
    \begin{definition}
        Let $X$ be a set, $\mathcal{R}$ a ring, and a measure $\mu \colon \mathcal{R} \to [0, \infty]$. Then, we define $\mu^* \colon \mathcal{P}(X) \to [0, \infty]$ by:
        \[\mu^*(A) = \inf \left\{\sum_{j=1}^\infty \mu(E_j) \mid (E_j)_{j=1}^\infty \text{ in } \mathcal{R}, A \subseteq \bigcup_{j=1}^\infty E_j\right\}\]
        and $\mu^*(A) = \infty$ if there is no $(E_j)_{j=1}^\infty$ in $\mathcal{R}$ containing $A$.
    \end{definition}

    \begin{lemma}
        Let $X$ be a set, $\mathcal{R}$ a ring, and a measure $\mu \colon \mathcal{R} \to [0, \infty]$. Then,
        \begin{enumerate}
            \item $\mu^*(\varnothing) = 0$;
            \item for $A \subseteq B \subseteq X$, $\mu^*(A) \leq \mu^*(B)$;
            \item for all $A \in \mathcal{R}$, $\mu^*(A) = \mu(A)$;
            \item for a sequence $(A_n)_{n=1}^\infty$ in $X$,
            \[\mu^* \left(\bigcup_{n=1}^\infty A_n\right) \leq \sum_{n=1}^\infty \mu^* (A_n).\]
        \end{enumerate}
    \end{lemma}
    \begin{proof}
        \hspace{0pt}
        \begin{enumerate}
            \item Define the sequence $(E_j)_{j=1}^\infty$ in $\mathcal{R}$ by $E_j = \varnothing$. We know that
            \[\varnothing \subseteq \bigcup_{j=1}^\infty E_j,\]
            meaning that 
            \[\mu^*(\varnothing) \leq \sum_{j=1}^\infty \mu(E_j) = 0.\]
            We also know that $\mu(E) \geq 0$ for all $E \in \mathcal{R}$, meaning that $\mu^*(A) \geq 0$ for all $A \subseteq X$. Hence, $\mu^*(\varnothing) = 0$.

            \item Let $(E_j)_{j=1}^\infty$ be a sequence in $\mathcal{R}$ such that
            \[B \subseteq \bigcup_{j=1}^\infty E_j.\]
            In that case,
            \[A \subseteq \bigcup_{j=1}^\infty E_j\]
            as well. Hence, by the infimum property, we find that $\mu^*(A) \leq \mu^*(B)$.
            
            \item Define the sequence $(E_j)_{j=1}^\infty$ in $\mathcal{R}$ by $E_1 = A$ and $E_n = \varnothing$ for $n \geq 2$. Then,
            \[A \subseteq \bigcup_{j=1}^\infty E_j,\]
            meaning that $\mu^*(A) \leq \mu(A)$. 

            Now, let $(E_j)_{j=1}^\infty$ in $\mathcal{R}$ such that
            \[A \subseteq \bigcup_{j=1}^\infty E_j.\]
            In that case,
            \[\mu(A) \leq \mu \left(\bigcup_{j=1}^\infty E_j\right).\]
            Hence, $\mu(A) \leq \mu^*(A)$. This implies that $\mu(A) = \mu^*(A)$.
            
            \item 
            % TODO
        \end{enumerate}
    \end{proof}

    \begin{definition}[Caratheodory's Condition]
        Let $X$ be a set, $\mathcal{R}$ a ring, a measure $\mu \colon \mathcal{R} \to [0, \infty]$, and $A \subseteq X$. We say that $A$ is \emph{$\mu^*$-measurable} if for all $S \subseteq X$,
        \[\mu^*(S) = \mu^*(S \cap A) + \mu^*(S \cap A^c).\]
        We denote by $\mathcal{M}_{\mu^*}$ the set of $\mu^*$-measurable sets of $X$.
    \end{definition}

    \begin{proposition}
        Let $X$ be a set, $\mathcal{R}$ a ring, measure $\mu \colon \mathcal{R} \to [0, \infty]$, and $A \subseteq X$. Then, for all $S \subseteq X$, 
        \[\mu^*(S \cap A) + \mu^*(S \cap A^c) \geq \mu^*(S).\]
    \end{proposition}
    \begin{proof}
        Define the sequence $(E_j)_{j=1}^\infty$ in $\mathcal{P}(X)$ by $E_1 = S \cap A$, $E_2 = S \cap A^c$, $E_n = \varnothing$ for $n \geq 3$. Then,
        \[\mu^*(S) = \mu^* \left(\bigcup_{n=1}^\infty E_n\right) \leq \sum_{n=1}^\infty \mu^*(E_n) = \mu^*(S \cap A) + \mu^*(S \cap A^c).\]
    \end{proof}

    \begin{proposition}
        Let $X$ be a set, $\mathcal{R}$ a ring, and a measure $\mu \colon \mathcal{R} \to [0, \infty]$. Then,
        \begin{enumerate}
            \item $\mathcal{R} \subseteq \mathcal{M}_{\mu^*}$;
            \item $\mathcal{M}_{\mu^*}$ is an algebra;
            \item $\mathcal{M}_{\mu^*}$ is a $\sigma$-algebra;
            \item $\mu^*$ is a measure on $\mathcal{M}_{\mu^*}$.
        \end{enumerate}
    \end{proposition}
    \begin{proof}
        \hspace{0pt}
        \begin{enumerate}
            \item Let $A \in \mathcal{R}$, $S \subseteq X$ and let $\varepsilon > 0$. We show that
            \[\mu^*(S) > \mu^*(S \cap A) + \mu^*(S \cap A^c) - \varepsilon.\]
            By definition of infimum, we can find a sequence $(E_j)_{j=1}^\infty$ in $\mathcal{R}$ with 
            \[S \subseteq \bigcup_{j=1}^\infty E_j \qquad \textrm{ s.t. } \qquad \mu^*(S) \leq \sum_{j=1}^\infty \mu(E_j) < \mu^*(S) + \varepsilon.\]
            We know that 
            \[S \cap A \subseteq \bigcup_{j=1}^\infty (E_j \cap A), \qquad S \cap A^c \subseteq \bigcup_{j=1}^\infty (E_j \cap A^c).\]
            Since rings are closed under intersection, we find that
            \begin{align*}
                \mu^*(S \cap A) + \mu^*(S \cap A^c) &\leq \sum_{j=1}^\infty \mu(E_j \cap A) + \sum_{j=1}^\infty \mu(E_j \cap A^c) \\
                &= \sum_{j=1}^\infty \mu(E_j) \\
                &< \mu^*(S) + \varepsilon.
            \end{align*}
            Hence,
            \[\mu^*(S) > \mu^*(S \cap A) + \mu^*(S \cap A^c) - \varepsilon\]
            for all $\varepsilon > 0$. So,
            \[\mu^*(S) \geq \mu^*(S \cap A) + \mu^*(S \cap A^c).\]
            This implies that $A \in \mathcal{M}_{\mu^*}$.

            \item \begin{itemize}
                \item For all $S \subseteq X$, we find that
                \[\mu^*(S \cap \varnothing) + \mu^*(S \cap \varnothing^c) = \mu^*(\varnothing) + \mu^*(S) = \mu^*(S).\]
                So, $\varnothing \in S$.

                \item Let $A \in \mathcal{M}_{\mu^*}$. In that case,
                \[\mu^*(S) = \mu^*(S \cap A) + \mu^*(S \cap A^c).\]
                Hence, $A^c \in \mathcal{M}_{\mu^*}$.

                \item Let $A, B \in \mathcal{M}_{\mu^*}$. For $S \subseteq X$, we find that
                \begin{align*}
                    \mu^*(S) &= \mu^*(S \cap A) + \mu^*(S \cap A^c) \\
                    &= \mu^*(S \cap A \cap B) + \mu^*(S \cap A \cap B^c) + \mu^*(S \cap A^c \cap B) \\
                    &+ \mu^*(S \cap A^c \cap B^c) \\
                    % TODO: Why 
                    &= \mu^*(S \cap (A \cup B)) + \mu^*(S \cap (A \cup B)^c).
                \end{align*}
                So, $A \cup B \in \mathcal{M}_{\mu^*}$.
            \end{itemize}

            \item Let $(A_n)_{n=1}^\infty$ be a sequence of disjoint sets in $\mathcal{M}_{\mu^*}$.
            % TODO
            
            \item 
            % TODO: Show additive function

            
        \end{enumerate}
    \end{proof}

    \begin{proposition}[Caratheodory Extension Theorem]
        Let $X$ be a set, $\mathcal{R}$ a ring, and a measure $\mu \colon \mathcal{R} \to [0, \infty]$. Then, $\mu$ extends to a measure on the $\sigma$-algebra $\mathcal{A}(\mathcal{R})$ generated by $\mathcal{R}$.
    \end{proposition}
    \begin{proof}
        
    \end{proof}

    \begin{proposition}
        The Lebesgue measure $\lambda \colon \mathcal{E}(\mathbb{R}) \to [0, \infty]$ extends to a unique measure $\lambda^* \colon \mathcal{B}(\mathbb{R}) \to [0, \infty]$.
    \end{proposition}
    \begin{proof}
        
    \end{proof}

    \begin{proposition}
        Let $x \in \mathbb{R}$ and $A \in \mathcal{B}(\mathbb{R})$. Then, 
        \[\lambda(x + A) = \lambda(A).\]
    \end{proposition}
    \begin{proof}
        % Let
        % \[\mathcal{M} = \{A \in \mathcal{B}(\mathbb{R}) \mid \lambda(x + A) = \lambda(A) \ \forall x \in \mathbb{R}\}.\]
        % We show that $\mathcal{M}$ is a $\sigma$-algebra. 
        % \begin{itemize}
        %     \item We have $\varnothing \in \mathcal{B}(\mathbb{R})$ since $\varnothing + A = \varnothing$.
            
        %     \item Let $A \in \mathcal{B}(\mathbb{R})$. In that case, $\lambda(x + A) = \lambda(A)$ for all $x \in \mathbb{R}$. Hence,
        %     \[\lambda(x + A^c) = \]
        % \end{itemize}
    \end{proof}
    
    \newpage

    \section{Probability and Independence}
    \begin{definition}
        Let $(\Omega, \mathcal{A}, P)$ be a measure space. We say that it is a \emph{probability space} if $P(\Omega) = 1$.
    \end{definition}

    \begin{definition}
        Let $(\Omega, \mathcal{A}, P)$ be a probability space, and let $(\mathcal{A}_i)_{i \in I}$ be a sequence of $\sigma$-algebras in $\mathcal{A}$, for some indexing set $I$. We say that $(\mathcal{A}_i)$ is independent if for any $J \subseteq I$ finite,
        \[P\left(\bigcap_{j \in J} A_j \right) = \prod_{j \in J} P(A_j).\]
    \end{definition}

    \begin{definition}
        Let $(A_n)_{n=1}^\infty$ be a sequence of events in some probability space. Define
        \[\limsup A_n = \bigcap_{n=1}^\infty \bigcup_{m=n}^\infty A_m, \qquad \liminf A_n = \bigcup_{n=1}^\infty \bigcap_{m=n}^\infty A_m.\]
    \end{definition}

    \begin{proposition}
        Let $(A_n)_{n=1}^\infty$ be a sequence of events in some probability space. Then,
        \begin{itemize}
            \item $(\limsup A_n)^c = \liminf A_n^c$;
            \item $(\liminf A_n)^c = \limsup A_n^c$;
            \item $\liminf A_n \subseteq \limsup A_n$.
        \end{itemize}
    \end{proposition}
    \begin{proof}
        
    \end{proof}

    \begin{lemma}[First Borel-Cantelli Lemma]
        Let $X$ be a set, $\mathcal{R}$ a ring of subsets of $X$, $P$ a probability measure on $X$. Then, for a sequence $(A_n)_{n=1}^\infty$ in $\mathcal{R}$ with 
        \[\sum_{n=1}^\infty P(A_n)\]
        finite, $P(\liminf A_n) = 0$.
    \end{lemma}
    \begin{proof}
        Define the sequence $(B_n)_{n=1}^\infty$ in $\mathcal{R}$ by
        \[B_n = \bigcup_{m=n}^\infty A_m.\]
        We have $B_n \supseteq B_{n+1}$ for all $n \in \mathbb{Z}_{\geq 1}$ with
        \[\bigcap_{n=1}^\infty B_n = \varnothing.\]
        Since $P$ is a measure, this implies that
        \[P(\liminf A_n) = P \left(\bigcap_{n=1}^\infty B_n\right) = \lim_{n \to \infty} P(B_n).\]
        We also know that
        \[P(B_n) = P \left(\bigcup_{m=n}^\infty A_m\right) \leq \sum_{m=n}^\infty A_m \to 0\]
        by assumption. Hence, $P(\liminf A_n) = 0$.
    \end{proof}

    \begin{lemma}[Second Borel-Cantelli Lemma]
        Let $X$ be a set, $\mathcal{R}$ a ring of subsets of $X$, $P$ a probability measure on $X$. Then, for a sequence of independent events $(A_n)_{n=1}^\infty$ in $\mathcal{R}$ with
        \[\sum_{n=1}^\infty P(A_n) = \infty,\]
        $P(\limsup A_n) = 1$.
    \end{lemma}
    \begin{proof}
        We know that for all $a \geq 0$, $1 - a \leq e^{-a}$. Now, let $a_n = P(A_n)$. For $N, n \in \mathbb{Z}_{\geq 1}$ with $N \geq n$, we have
        \[P \left(\bigcap_{m=n}^N A_m^c\right) = \prod_{m=n}^N P(A_m^c)\]
        since $(A_n)$ is a sequence of independent events. Moreover,
        \[P(A_m^c) = (1 - a_m) \leq e^{-a_m},\]
        meaning that
        \[P \left(\bigcap_{m=n}^N A_m^c\right) \leq \exp \left(- \sum_{m=n}^N a_m\right) \to 0\]
        as $\sum_{m=n}^N a_m \to \infty$ as $N \to \infty$, by assumption. Hence,
        \[\lim_{N \to \infty} P \left(\bigcap_{m=n}^N P(A_n^c)\right) = P \left(\bigcap_{m=n}^\infty P(A_n^c)\right) = 0\]
        for all $n \in \mathbb{Z}_{\geq 1}$. So,
        \[P(\limsup A_n^c) = P \left(\bigcup_{n=1}^\infty \bigcap_{m=n}^\infty A_m^c\right) \leq \sum_{n=1}^\infty P \left(\bigcap_{m=n}^\infty A_m^c\right) = 0.\]
        Hence,
        \[P(\liminf A_n) = 1 - P(\limsup A_n^c) = 1 - 0 = 1.\]
    \end{proof}
\end{document}