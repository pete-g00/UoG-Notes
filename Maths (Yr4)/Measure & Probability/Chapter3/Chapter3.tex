\documentclass[a4paper, openany]{memoir}

\usepackage[utf8]{inputenc}
\usepackage[T1]{fontenc} 
\usepackage[english]{babel}

\usepackage{fancyhdr}
\usepackage{float}
\usepackage{bm}

\usepackage{amsmath}
\usepackage{amsthm}
\usepackage{amssymb}
\usepackage{enumitem}
\usepackage{multicol}
\usepackage[bookmarksopen=true,bookmarksopenlevel=2]{hyperref}
\usepackage{tikz}
\usepackage{indentfirst}

\pagestyle{fancy}
\fancyhf{}
\fancyhead[LE]{\leftmark}
\fancyhead[RO]{\rightmark}
\fancyhead[RE, LO]{Measure \& Probability}
\fancyfoot[LE, RO]{\thepage}
\fancyfoot[RE, LO]{Pete Gautam}

\renewcommand{\headrulewidth}{1.5pt}

\theoremstyle{definition}
\newtheorem{definition}{Definition}[section]
\newtheorem{example}[definition]{Example}

\theoremstyle{plain}
\newtheorem{theorem}[definition]{Theorem}
\newtheorem{lemma}[definition]{Lemma}
\newtheorem{proposition}[definition]{Proposition}
\newtheorem{corollary}[definition]{Corollary}

\chapterstyle{thatcher}

\setcounter{chapter}{2}

\begin{document}
    \chapter{Product Measures}
    \section{Product Algebras}
    \begin{definition}
        Let $(X, \mathcal{A}, \mu)$ and $(Y, \mathcal{B}, \nu)$ be measure spaces. We define the \emph{product $\sigma$-algebra} $\mathcal{A} \otimes \mathcal{B}$ by the $\sigma$-algebra generated by sets of the form $A \times B$, where $A \in \mathcal{A}$ and $B \in \mathcal{B}$.
    \end{definition}

    \begin{definition}
        Let $X$ and $Y$ be sets and let $E \subseteq X \times Y$. For $x \in X$, we define the \emph{$x$-section of $E$} by the set
        \[E_x = \{y \in Y \mid (x, y) \in E\},\]
        and for $y \in Y$, the \emph{$y$-section of $E$} by the set
        \[E^y = \{x \in X \mid (x, y) \in E\}.\]
    \end{definition}

    \begin{lemma}
        Let $X$ and $Y$ be sets and let $E \subseteq X \times Y$. Then, 
        \begin{enumerate}
            \item for $x \in X$ and $y \in Y$, $(E_x)^c = (E^c)_x$ and $(E^y)^c = (E^c)^y$;
            \item for a sequence of subsets $(E_n)_{n=1}^\infty$ in $X \times Y$,
            \begin{align*}
                \left(\bigcup_{n=1}^\infty E_n \right)_x &= \bigcup_{n=1}^\infty (E_n)_x & \left(\bigcap_{n=1}^\infty E_n \right)_x &= \bigcap_{n=1}^\infty (E_n)_x \\
                \left(\bigcup_{n=1}^\infty E_n \right)^y &= \bigcup_{n=1}^\infty (E_n)^y & \left(\bigcap_{n=1}^\infty E_n \right)^y &= \bigcap_{n=1}^\infty (E_n)^y.
            \end{align*}
        \end{enumerate}
    \end{lemma}
    \begin{proof}
        \hspace*{0pt}
        \begin{enumerate}
            \item Let $x \in X$. For $y \in Y$, we have
            \begin{align*}
                y \in (E_x)^c &\iff y \not\in E_x \\
                &\iff (x, y) \not\in E \\
                &\iff (x, y) \in E^c \\
                &\iff y \in (E^c)_x.
            \end{align*}
            Hence, $(E_x)^c = (E^c)_x$. Similarly, $(E^y)^c = (E^c)^y$.

            \item Let $x \in X$. For $y \in Y$, we have
            \begin{align*}
                y \in \left(\bigcup_{n=1}^\infty E_n\right)_x &\iff \exists n \in \mathbb{Z}_{\geq 1} \textrm{ s.t. } (x, y) \in E_n \\
                &\iff \exists n \in \mathbb{Z}_{\geq 1} \textrm{ s.t.} y \in (E_n)_x \\
                &\iff y \in \bigcup_{n=1}^\infty (E_n)_x.
            \end{align*}
            Also, for $y \in Y$, we have
            \begin{align*}
                y \in \left(\bigcap_{n=1}^\infty E_n\right)_x &\iff \forall n \in \mathbb{Z}_{\geq 1}, (x, y) \in E_n \\
                &\iff \forall n \in \mathbb{Z}_{\geq 1}, y \in (E_n)_x \\
                &\iff y \in \bigcap_{n=1}^\infty (E_n)_x.
            \end{align*}
            Hence,
            \[\left(\bigcup_{n=1}^\infty E_n \right)_x = \bigcup_{n=1}^\infty (E_n)_x \qquad \left(\bigcap_{n=1}^\infty E_n \right)_x = \bigcap_{n=1}^\infty (E_n)_x.\]
            Similarly,
            \[\left(\bigcup_{n=1}^\infty E_n \right)^y = \bigcup_{n=1}^\infty (E_n)^y \qquad
            \left(\bigcap_{n=1}^\infty E_n \right)^y = \bigcap_{n=1}^\infty (E_n)^y.\]
        \end{enumerate}
    \end{proof}

    \begin{proposition}
        Let $(X, \mathcal{A}, \mu)$ and $(Y, \mathcal{B}, \nu)$ be measure spaces and $E \in \mathcal{A} \otimes \mathcal{B}$. Then, for all $x \in X$ and $y \in Y$, $E_x \in \mathcal{B}$ and $E^y \in \mathcal{A}$.
    \end{proposition}
    \begin{proof}
        Let 
        \[\mathcal{M} = \{S \in \mathcal{A} \otimes \mathcal{B} \mid S_x \in \mathcal{B} \ \forall x \in X, S^y \in \mathcal{A} \ \forall y \in Y\}.\]
        We show that $\mathcal{M}$ is a $\sigma$-algebra.
        \begin{itemize}
            \item We have $\varnothing \in \mathcal{M}$ since $\varnothing_x = \varnothing \in \mathcal{B}$ and $\varnothing^y = \varnothing \in \mathcal{A}$ for all $x \in X$ and $y \in Y$.
            
            \item Let $S \in \mathcal{M}$. Then, for all $x \in X$ and $y \in Y$,
            \[(S^c)_x = (S_x)^c \in \mathcal{B}, \qquad (S^c)^y = (S^y)^c \in \mathcal{A}\]
            since $\mathcal{A}$ and $\mathcal{B}$ are $\sigma$-algebras.
            
            \item Let $(S_n)_{n=1}^\infty$ be a sequence in $\mathcal{M}$. Then, for all $x \in X$ and $y \in Y$,
            \[\left(\bigcup_{n=1}^\infty S_n\right)_x = \bigcup_{n=1}^\infty (S_n)_x \in \mathcal{B}, \qquad \left(\bigcup_{n=1}^\infty S_n\right)^y = \bigcup_{n=1}^\infty (S_n)^y \in \mathcal{A}\]
            since $\mathcal{A}$ and $\mathcal{B}$ are $\sigma$-algebras.
        \end{itemize}
        Hence, $\mathcal{M}$ is a $\sigma$-algebra. Now, let $A \in \mathcal{A}$ and $B \in \mathcal{B}$. For $x \in X$ and $y \in Y$, we have
        \[(A \times B)_x = \begin{cases}
            B & x \in A \\
            \varnothing & \textrm{otherwise},
        \end{cases} \qquad (A \times B)^y = \begin{cases}
            A & y \in B \\ 
            \varnothing & \textrm{otherwise}.
        \end{cases}\]
        So, $A \times B \in \mathcal{M}$. Since $\mathcal{A} \otimes \mathcal{B}$ is generated by $\mathcal{A} \times \mathcal{B}$, we find that $\mathcal{M} = \mathcal{A} \times \mathcal{B}$. That is, for all $E \in \mathcal{A} \otimes \mathcal{B}$, $E_x \in \mathcal{B}$ and $E^y \in \mathcal{A}$.
    \end{proof}

    \begin{definition}
        Let $X$ and $Y$ be sets and let $f \colon X \times Y \to \mathbb{R} \cup \{\pm \infty\}$ be a function. For $x \in X$, define the function $f_x \colon Y \to \mathbb{R} \cup \{\pm \infty\}$ by $f_x(y) = f(x, y)$, and for $y \in Y$, define the function $f_y \colon X \to \mathbb{R} \cup \{\pm \infty\}$ by $f_y(x) = f(x, y)$.
    \end{definition}

    \begin{lemma}
        Let $X$ and $Y$ be sets and let $f \colon X \times Y \to \mathbb{R} \cup \{\pm \infty\}$ be a function. Then, for $a \in \mathbb{R}$, $x \in X$ and $y \in Y$,
        \[(f_x)^{-1}(a, \infty] = (f^{-1}(a, \infty])_x, \qquad (f_y)^{-1}(a, \infty] = (f^{-1}(a, \infty])_y.\]
    \end{lemma}
    \begin{proof}
        Let $a \in \mathbb{R}$ and $x \in X$. For $y \in Y$, we find that
        \begin{align*}
            y \in (f_x)^{-1}(a, \infty] &\iff f(x, y) = f_x(y) \in (a, \infty] \\
            &\iff (x, y) \in f^{-1}(a, \infty] \\
            &\iff y \in (f^{-1}(a, \infty])_x.
        \end{align*}
        Hence, $(f_x)^{-1}(a, \infty] = (f^{-1}(a, \infty])_x$ for all $x \in X$. Similarly, $(f_y)^{-1}(a, \infty] = (f^{-1}(a, \infty])^y$ for all $y \in Y$.
    \end{proof}

    \begin{proposition}
        Let $(X, \mathcal{A}, \mu)$ and $(Y, \mathcal{B}, \nu)$ be measure spaces and $f \colon X \times Y \to \mathbb{R} \cup \{\pm \infty\}$ be measurable with respect to $\mathcal{A} \otimes \mathcal{B}$. Then, for all $x \in X$, $f_x$ is measurable with respect to $\mathcal{B}$ and for all $y \in Y$, $f_y$ is measurable with respect to $\mathcal{A}$.
    \end{proposition}
    \begin{proof}
        Let $a \in \mathbb{R}$. Since $f$ is measurable, we find that
        \[S = f^{-1}(a, \infty] \in \mathcal{A} \otimes \mathcal{B}.\]
        Now, let $x \in X$ and $y \in Y$. We have shown above that $S_x \in \mathcal{B}$ and $S^y \in \mathcal{A}$. In that case,
        \[(f_x)^{-1}(a, \infty] = S_x \in \mathcal{B}, \qquad (f_y)^{-1}(a, \infty] = S^y \in \mathcal{A}.\]
        This implies that $f_x$ is measurable with respect to $\mathcal{B}$, and $f_y$ is measurable with respect to $\mathcal{A}$.
    \end{proof}

    \begin{definition}
        Let $(X, \mathcal{A}, \mu)$ and $(Y, \mathcal{B}, \nu)$ be measure spaces. Define the set $\mathcal{E}$ by the elements in $X \times B$ that are a finite union of elements in $\mathcal{A} \times \mathcal{B}$.
    \end{definition}

    \begin{proposition}
        Let $(X, \mathcal{A}, \mu)$ and $(Y, \mathcal{B}, \nu)$ be measure spaces. Then, $\mathcal{E}$ is an algebra.
    \end{proposition}
    \begin{proof}
        \hspace*{0pt}
        \begin{itemize}
            \item We have $\varnothing \in \mathcal{E}$ since $\varnothing \in \mathcal{A} \times \mathcal{B}$.
            
            \item Let $E, F \in \mathcal{E}$. Then, $E$ and $F$ are both finite union of elements in $\mathcal{A} \times \mathcal{B}$. Hence, their union $E \times F$ is a finite union of elements in $\mathcal{A}$.
            
            \item Let $A \in \mathcal{A}$ and $B \in \mathcal{B}$. For $(x, y) \in X \times Y$, we have 
            \begin{align*}
                (x, y) \in (A \times B)^c &\iff (x, y) \not\in A \times B \\
                &\iff x \not\in A \textrm{ or } y \not\in B \\
                &\iff (x \in A^c \textrm{ and } y \in B) \textrm{ or } (x \in A^c \textrm{ and } y \in B^c) \\
                & \textrm{ or } (x \in A \textrm{ and } y \in B^c) \\
                &\iff (x, y) \in (A^c \times B) \cup (A \times B^c) \cup (A^c \times B^c).
            \end{align*}
            Hence,
            \[(A \times B)^c = (A^c \times B) \cup (A \times B^c) \times (A^c \times B^c),\]
            where each of the 3 subsets is disjoint. This implies that $(A \times B)^c \in \mathcal{E}$. We know that $\mathcal{E}$ is closed under the union of finite intervals from above, so in general, for all $E \in \mathcal{E}$, $E^c \in \mathcal{E}$.
        \end{itemize}
        So, $\mathcal{E}$ is an algebra.
    \end{proof}

    \begin{proposition}
        Let $(X, \mathcal{A}, \mu)$ and $(Y, \mathcal{B}, \nu)$ be measure spaces. Define the map $\mu \otimes \nu \colon \mathcal{A} \times \mathcal{B} \to [0, \infty]$ by
        \[(\mu \otimes \nu) \left(\bigcup_{k=1}^n A_k \times B_k\right) = \sum_{k=1}^n \mu(A_k) \cdot \mu(B_k)\]
        whenever $(A_k \times B_k)_{k=1}^n$ is pairwise disjoint. Then, $\mu \otimes \nu$ is a measure.
    \end{proposition}
    \begin{proof}
        \hspace*{0pt}
        \begin{itemize}
            \item We first show that $\mu \otimes \nu$ is well-defined. So, let $(A_i \times B_i)_{i=1}^n$ and $(C_j \times D_j)_{j=1}^m$ be disjoint sequences of sets in $\mathcal{A} \times \mathcal{B}$ such that
            \[\bigcup_{i=1}^n A_i \times B_i = \bigcup_{j=1}^m C_j \times D_j.\]
        \end{itemize}
    \end{proof}

    \begin{proposition}
        Let $(X, \mathcal{A}, \mu)$ and $(Y, \mathcal{B}, \nu)$ be measure spaces. Then, there exists a measurable functions $\mu \otimes \nu$ on $\mathcal{A} \otimes \mathcal{B}$ such that for all $A \in \mathcal{A}$ and $B \in \mathcal{B}$,
        \[(\mu \otimes \nu)(A \otimes B) = \mu(A) \cdot \nu(B).\]
    \end{proposition}
    \begin{proof}
        By Caratheodory Extension Theorem, we know that the measure $\mu \otimes \nu$ extends to the $\sigma$-algebra generated by $\mathcal{A} \times \mathcal{B}$, i.e. $\mathcal{A} \otimes \mathcal{B}$.
    \end{proof}

    \begin{proposition}
        Let $(X, \mathcal{A}, \mu)$ and $(Y, \mathcal{B}, \nu)$ be measure spaces and $A \in \mathcal{A}$, $B \in \mathcal{B}$. If there exists a sequence of disjoint subsets $(A_n \times B_n)_{n=1}^\infty$ such that
        \[A \times B = \bigcup_{n=1}^\infty A_n \times B_n,\]
        then
        \[(\mu \otimes \nu)(A \otimes B) = \mu(A) \cdot \nu(B) = \sum_{n=1}^\infty (\mu \otimes \nu) (A_n \otimes B_n).\]
    \end{proposition}
    \begin{proof}
        
    \end{proof}

    % \begin{proposition}
    %     Let $(X, \mathcal{A}, \mu)$ and $(Y, \mathcal{B}, \nu)$ be measure spaces. Define the set $\mathcal{E} \subseteq \mathcal{A} \otimes \mathcal{B}$ by sets of the form
    %     \[\bigcup_{i=1}^k A_i \times B_i,\]
    %     where $k \in \mathbb{Z}_{\geq 1}$, $A_i \in \mathcal{A}$, $B_i \in \mathcal{B}$ for $1 \leq i \leq k$ with $(A_i \times B_i) \cap (A_j \times B_j) = \varnothing$ if $i \neq j$. Then, the function $\mu \otimes \nu$ is a measure on $\mathcal{E}$.
    % \end{proposition}
    % \begin{proof}
        
    % \end{proof}

    \begin{lemma}
        Let $(X, \mathcal{A}, \mu)$ and $(Y, \mathcal{B}, \nu)$ be $\sigma$-finite measure spaces and $E \in \mathcal{A} \otimes \mathcal{B}$. Then, the functions $f \colon X \to [0, \infty]$ and $g \colon Y \to [0, \infty]$ defined by $f(x) = \nu(E_x)$ and $g(x) = \mu(E_y)$ are measurable with
        \[(\mu \otimes \nu)(E) = \int_X f \ d\mu = \int_Y g \ d\nu,\]
        meaning that
        \[(\mu \otimes \nu)(E) = \int_X \nu(E_x) \ d\mu(x) = \int_Y \mu(E^y) \ d\nu(y).\]
    \end{lemma}
    \begin{proof}
        Let
        \[\mathcal{M} = \left\{E \in \mathcal{A} \otimes \mathcal{B} \mid (\mu \otimes \nu)(E) = \int_X \nu(E_x) \ d\nu(x) = \int_Y \mu(E^y) \ d\mu(y)\right\}.\]
        We claim that $\mathcal{M}$ is a $\sigma$-algebra containing $\mathcal{A} \times \mathcal{B}$.
        \begin{itemize}
            \item We have
            \begin{align*}
                \mu(\varnothing) &= 0 \\
                \int_X \nu(\varnothing_x) \ d\mu(x) = \int_X 0 \ d\mu(x) &= 0 \\
                \int_Y \mu(\varnothing_y) \ d\nu(y) = \int_Y 0 \ d\nu(y) &= 0.
            \end{align*}
            So, $\varnothing \in \mathcal{M}$.

            \item Let $(E_n)_{n=1}^\infty$ be a sequence of disjoint sets in $\mathcal{M}$. We have
            \[\int_X \nu \left(\bigcup_{n=1}^\infty (E_n)_x\right) \ d\mu(x) = \int_X \sum_{n=1}^\infty \nu((E_n)_x) \ d\mu(x)\]
            since $\nu$ is a measure. Since $\nu((E_n)_x) \geq 0$, Monotone Convergence Theorem tells us that
            \[\int_X \sum_{n=1}^\infty \nu((E_n)_x) \ d\mu(x) = \sum_{n=1}^\infty \int_X \nu((E_n)_x) \ d\mu(x).\]
            Since $E_n \in \mathcal{M}$ for all $n \in \mathbb{Z}_{\geq 1}$,
            \[\sum_{n=1}^\infty \int_X \nu ((E_n)_x) \ d\mu(x) = \sum_{n=1}^\infty (\mu \otimes \nu)((E_n)_x).\]
            Now, since $\mu \otimes \nu$ is a measure, we find that
            \[\sum_{n=1}^\infty (\mu \otimes \nu) ((E_n)_x) = (\mu \otimes \nu) (\bigcup_{n=1}^\infty E_n)_x.\]
            Hence,
            \[\int_X \nu \left(\bigcup_{n=1}^\infty (E_n)_x\right) \ d\mu(x) = (\mu \otimes \nu) (\bigcup_{n=1}^\infty E_n)_x.\]
            Similarly,
            \[\int_Y \mu \left(\bigcup_{n=1}^\infty (E_n)^y \right) \ d\nu(y) = (\mu \otimes \nu) (\bigcup_{n=1}^\infty E_n)^y.\]
            This implies that
            \[\bigcup_{n=1}^\infty E_n \in \mathcal{M}.\]
            
            \item By Dynkin's Lemma, we find that $\mathcal{M}$ is closed under complements.
            
            \item Now, let $A \in \mathcal{A}$ and $B \in \mathcal{B}$, and denote $E = A \times B$. We know that for all $x \in X$,
            \[\nu(E_x) = \begin{cases}
                \nu(B) & x \in A \\
                0 & x \in B.
            \end{cases}\]
            Hence,
            \[\int_X \nu (E_x) \ d\mu(x) = \int_X \chi_A \nu(B) \ d\mu(x) = \nu(B) \cdot \mu(X) = (\mu \otimes \nu)(A \times B).\]
            Similarly,
            \[\int_Y \mu(E^y) \ d\nu(y) = \int_Y \chi_B \mu(A) \ d\nu(y) = (\mu \otimes \nu)(A \times B).\]
            This implies that $E \in \mathcal{M}$.
        \end{itemize}
        So, since $\mathcal{M}$ is a $\sigma$-algebra containing $\mathcal{A} \times \mathcal{B}$, we find that $\mathcal{M} = \mathcal{A} \otimes \mathcal{B}$. Hence, 
        \[(\mu \otimes \nu)(E) = \int_X \nu(E_x) \ d\mu(x) = \int_Y \mu(E^y) \ d\nu(y)\]
        for all $E \in \mathcal{A} \otimes \mathcal{B}$.
    \end{proof}

    \begin{theorem}[Tonelli's Theorem]
        Let $(X, \mathcal{A}, \mu)$ and $(Y, \mathcal{B}, \nu)$ be $\sigma$-finite measure spaces and let $f \colon X \times Y \to [0, \infty]$ be integrable with respect to $\mu \otimes \nu$. Then, for $F \colon X \to [0, \infty]$ and $G \colon Y \to [0, \infty]$ given by
        \[F(x) = \int_Y f(x, y) \ d\nu(y), \qquad G(y) = \int_X f(x, y) \ d\mu(x)\]
        are $\nu$- and $\mu$-measurable respectively. Then, $f$ is integrable,
        \[\int_{X \times Y} f \ d(\mu \otimes \nu) = \int_X F \ d\mu = \int_Y G \ d\nu.\]
    \end{theorem}
    \begin{proof}
        Let $E \in \mathcal{A} \otimes \mathcal{B}$. We first show that $f = \chi_E$ satisfies the result. For $x \in X$ and $y \in Y$, we have
        \[(\chi_E)_x(y) = \chi_E(x, y) = \begin{cases}
            1 & (x, y) \in E \\
            0 & \textrm{otherwise}
        \end{cases} = \chi_{E_x}(y),\]
        so $(\chi_E)_x = \chi_{E_x}$. Similarly, $(\chi_E)^y = \chi_{E^y}$. This implies that
        \[F(x) = \int_Y f_x \ d\nu(y) = \int_Y \chi_{E_x} \ d\nu = \nu(E_x),\]
        and
        \[G(y) = \int_X f^y \ d\mu(y) = \int_X \chi_{E^y} \ d\mu = \mu(E^y).\]
        By the result above, we know that
        \begin{align*}
            \int_X F \ d\mu = \int_X \nu(E_x) \ d\mu(x) &= (\mu \otimes \nu)(E) \\
            \int_Y G \ d\nu = \int_Y \mu(E^y) \ d\nu(y) &= (\mu \otimes \nu)(E).
        \end{align*}
        Now, the result follows for a simple measurable function- it is a linear combination of characteristic functions on measurable sets.

        Next, let $f \geq 0$ be measurable. We know that there exists a sequence of (monotone) simple functions $(s_n)_{n=1}^\infty$ with $s_n \to f$. We find that
        \begin{align*}
            \int_Y F_n(x) \ d\nu(y) &\to \int_Y F \ d\nu(y) \\
            \int_X G_n(y) \ d\mu(x) &\to \int_X G \ d\mu(x)
        \end{align*}
        by Monotone Convergence Theorem. Again, by MCT,
        \begin{align*}
            \int_X F_n \ d\mu(x) &= \int_X \int_Y s_n \ d\mu \otimes \nu \to \int_X F(x) \ d\mu(x) \\
            \int_X G_n \ d\mu(x) &= \int_Y \int_X s_n \ d\mu \otimes \nu \to \int_Y G(y) \ d\nu(y).
        \end{align*}
        Since
        \[\int_X \int_Y s_n \ d\mu \otimes \nu = \int_Y \int_X s_n \ d\mu \otimes d\nu,\]
        we find that
        \[\int_X F \ d\mu(x) = \int_Y G \ d\nu(y) = \int_{X \times Y} f \ d\mu \otimes d\nu.\]
        % TODO: Needs more justification
    \end{proof}

    \begin{theorem}[Fubini's Theorem]
        Let $(X, \mathcal{A}, \mu)$ and $(Y, \mathcal{B}, \nu)$ be $\sigma$-finite measure spaces and let $f \colon X \times Y \to \mathbb{R} \cup \{\pm \infty\}$ be integrable with respect to $\mu \otimes \nu$. Then,
        \begin{enumerate}
            \item for $\mu$-almost all $x$, the function $f_x \colon Y \to \mathbb{R} \cup \{\pm \infty\}$ given by $f_x(y) = f(x, y)$ is $\nu$ integrable; for $\nu$-almost all $y$, the function $f_y \colon X \to \mathbb{R} \cup \{\pm \infty\}$ given by $f_y(x) = f(x, y)$ is $\mu$ integrable.
            \item for $F \colon X \to \mathbb{R} \cup \{\pm \infty\}$ and $G \colon Y \to \mathbb{R} \cup \{\pm \infty\}$ defined by
            \[F(x) = \int_Y f(x, y) \ d\nu(y), \qquad G(y) = \int_X f(x, y) \ d\mu(x),\]
            $F$ is $\mu$-integrable and $G$ is $\nu$-integrable.
            \item \[\int_{X \times Y} f \ d(\mu \otimes \nu) = \int_X F \ d\mu = \int_Y G \ d\nu.\]
        \end{enumerate}
    \end{theorem}
    \begin{proof}
        This follows from Tonelli's Theorem, taking $f = f_+ + f_-$.
    \end{proof}
\end{document}