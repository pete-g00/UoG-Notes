\documentclass[a4paper, openany]{memoir}

\usepackage[utf8]{inputenc}
\usepackage[T1]{fontenc} 
\usepackage[english]{babel}

\usepackage{fancyhdr}
\usepackage{float}
\usepackage{bm}

\usepackage{amsmath}
\usepackage{amsthm}
\usepackage{amssymb}
\usepackage{enumitem}
\usepackage{multicol}
\usepackage[bookmarksopen=true,bookmarksopenlevel=2]{hyperref}
\usepackage{tikz}
\usepackage{indentfirst}

\pagestyle{fancy}
\fancyhf{}
\fancyhead[LE]{\leftmark}
\fancyhead[RO]{\rightmark}
\fancyhead[RE, LO]{Galois Theory}
\fancyfoot[LE, RO]{\thepage}
\fancyfoot[RE, LO]{Pete Gautam}

\renewcommand{\headrulewidth}{1.5pt}

\theoremstyle{definition}
\newtheorem{definition}{Definition}[section]
\newtheorem{example}[definition]{Example}

\theoremstyle{plain}
\newtheorem{theorem}[definition]{Theorem}
\newtheorem{lemma}[definition]{Lemma}
\newtheorem{proposition}[definition]{Proposition}
\newtheorem{corollary}[definition]{Corollary}

\setcounter{chapter}{1}

\chapterstyle{thatcher}

\begin{document}
    \chapter{Galois Extensions}
    \section{Field Extensions}
    \begin{definition}
        Let $K$ and $F$ be fields such that $K \subseteq F$ is a subring. We say that $K$ is a \emph{subfield} of $F$, and that $F$ is a \emph{field extension} of $K$, denoted $F|K$. If $K \subseteq E \subseteq F$ are fields, we call $E$ an \emph{intermediate field} of the field extension $F|K$.
    \end{definition}

    \begin{definition}
        Let $F|K$ be a field extension and let $\mathcal{S} \subseteq F$. We denote by $K(\mathcal{S}) \subseteq F$ the intermediate field of $F|K$ generated by $\mathcal{S}$. In particular, $K(\mathcal{S})$ is the intersection of all intermediate fields $K \subseteq E \subseteq F$ such that $\mathcal{S} \subseteq E$. If $\mathcal{S} = \{\alpha\}$, then we write $K(\alpha)$ instead of $K(\mathcal{S})$.
    \end{definition}

    \begin{definition}
        Let $F|K$ be a field extension. We define the \emph{degree of the field extension} $[F:K]$ by the dimension of the $K$-vector space $F$, i.e.
        \[[F : K] = \dim_K(F).\]
        We say that the field extension $F|K$ is finite if the degree $[F:K]$ is finite. Otherwise, $F|K$ is infinite.
    \end{definition}

    \begin{proposition}[The Tower Law]
        Let $K \subseteq E \subseteq F$ be field extensions. If $F|E$ and $E|K$ are finite, then $F|K$ is finite, with
        \[[F : K] = [F : E][E : K].\]
        Moreover, $[F : K]$ is infinite if and only if $[F : E]$ or $[E : K]$ is infinite.
    \end{proposition}
    \begin{proof}
        Assume that $F|E$ and $E|K$ are finite. Then, we can find a finite $K$-basis for $E$ and a finite $E$-basis $F$ respectively:
        \[\{e_1, e_2, \dots, e_n\}, \qquad \{f_1, f_2, \dots, f_m\}.\]
        We claim that the following is a $K$-basis for $F$:
        \[\{e_1f_1, e_1f_2, \dots, e_nf_m\}.\]
        So, let $\alpha \in F$. Using the $E$-basis for $F$, we can find $\alpha_1, \alpha_2, \dots, \alpha_m \in E$ such that
        \[\alpha = \alpha_1 f_1 + \alpha_2 f_2 + \dots + \alpha_m f_m.\]
        Now, using the $K$-basis for $E$, we can find $\beta_{i, j} \in K$ for $1 \leq i \leq m$ and $1 \leq j \leq n$ such that
        \begin{align*}
            \alpha &= \alpha_1 f_1 + \dots + \alpha_m f_m \\
            &= (\beta_{1, 1} e_1 + \dots + \beta_{1, n} e_n) f_1 + \dots + (\beta_{m, 1} e_1 + \dots + \beta_{m, n} e_n) f_m \\
            &= \beta_{1, 1} e_1f_1 + \dots + \beta_{1, n} e_nf_1 + \dots + \beta_{m, n} e_nf_m.
        \end{align*}
        Hence, the set 
        \[\{e_1f_1, e_1f_2, \dots, e_nf_m\}.\]
        spans $F$. Now, let $\alpha_{i, j} \in K$ for $1 \leq i \leq m$ and $1 \leq j \leq n$ such that
        \begin{align*}
            0 &= \alpha_{1, 1} e_1 f_1 + \dots + \alpha_{m, n} e_n f_m \\
            &= (\alpha_{1, 1} e_1 + \dots + \alpha_{1, n} e_n) f_1 + \dots (\alpha_{m, 1} e_1 + \dots + \alpha_{m, n} f_m).
        \end{align*}
        In that case, since $\{f_1, f_2, \dots, f_m\}$ forms a basis for $F$, we find that 
        \[\alpha_{i, 1} e_1 + \dots + \alpha_{i, n} e_n = 0\]
        for all $1 \leq i \leq m$. Moreover, since $\{e_1, e_2, \dots, e_n\}$ forms a basis for $E$, we find that $\alpha_{i, j} = 0$ for all $1 \leq i \leq m$ and $1 \leq j \leq n$. Hence, the set 
        \[\{e_1f_1, e_1f_2, \dots, e_nf_m\}.\]
        forms a basis for $F$. In particular, 
        \[[F : K] = m \cdot n = [F : E][E : K].\]
    \end{proof}

    \begin{definition}
        Let $F|K$ be a field extension, and let $\alpha \in F$. We say that $\alpha$ is \emph{algebraic} over $K$ if there exists a non-zero polynomial $f \in K[x]$ such that $f(\alpha) = 0$. Otherwise, $\alpha$ is \emph{transcendental} over $K$. We say that the extension $F|K$ is \emph{algebraic} if for all $\alpha \in F$, $\alpha$ is algebraic over $K$.
    \end{definition}

    \begin{proposition}
        Let $F|K$ be a finite extension. Then, $F|K$ is an algebraic extension.
    \end{proposition}
    \begin{proof}
        Let $[F:K] = n$, and let $\alpha \in F$. We know that the set
        \[\{1, \alpha, \alpha^2, \dots, \alpha^n\}\]
        has $n+1$ elements, so it cannot be linearly independent in $K$. Hence, there exist $k_0, k_1, \dots, k_n \in K$, not all zero, such that
        \[k_n \alpha^n + \dots + k_1 \alpha + k_0 = 0.\]
        So, define the polynomial $f \in K[x]$ by
        \[f(x) = k_n x^n + \dots + k_1 x + k_0.\]
        Since not all of $k_i$'s are zero for $0 \leq i \leq n$, we find that $f$ is a non-zero polynomial. Moreover, $f(\alpha) = 0$ by construction. Hence, $\alpha$ is algebraic over $K$, meaning that $F|K$ is an algebraic extension.
    \end{proof}

    \begin{lemma}
        Let $F|K$ be a field extension and let $\alpha \in F$. Then, $\alpha$ is algebraic over $K$ if and only if the evaluation map $ev_\alpha \colon K[x] \to K$ is not injective.
    \end{lemma}
    \begin{proof}
        Assume first that $\alpha$ is algebraic over $K$. In that case, there exists a non-zero polynomial $f \in K[x]$ such that $f(\alpha) = 0$. Hence, the evaluation map $ev_\alpha$ cannot be injective- there are 2 polynomials mapping to $0$.

        Now, assume that the evaluation map $ev_\alpha$ is not injective. In that case, the kernel $\ker ev_\alpha = (f)$ is non-zero, i.e. $f$ is non-zero. Moreover, $f(\alpha) = 0$, meaning that $\alpha$ is algebraic over $K$.
    \end{proof}

    \begin{proposition}
        Let $F|K$ be a field extension and let $\alpha \in F$ be algebraic over $K$. Then, there exists a unique monic polynomial $m_{\alpha, K} \in K[x]$, of smallest degree, such that
        \begin{itemize}
            \item $m_{\alpha, K} = 0$;
            \item $m_{\alpha, K}$ divides any $g \in K[x]$ with $g(\alpha) = 0$.
        \end{itemize}
    \end{proposition}
    \begin{proof}
        Since $\alpha$ is algebraic, the evaluation map $ev_\alpha \colon K[x] \to K$ is not injective. Moreover, since $K[x]$ is a principal ideal domain, we find that $\ker ev_\alpha = (f)$. Without loss of generality, assume that $f$ is monic. In that case, we can set $m_{\alpha, K} = f$, so that it satisfies both the properties.
    \end{proof}

    \begin{definition}
        Let $F|K$ be a field extension and let $\alpha \in F$ be algebraic over $K$. The unique monic polynomial of smallest degree $m_{\alpha, K} \in K[x]$ is called the \emph{minimal polynomial} of $\alpha$ over $K$.
    \end{definition}

    \begin{lemma}
        Let $F|K$ be a field extension and let $\alpha \in F$ be algebraic over $K$. The minimal polynomial $m_{\alpha, K}$ is irreducible. Moreover, it is the unique monic polynomial that is irreducible over $K$ such that $\alpha$ has a root.
    \end{lemma}
    \begin{proof}
        Let $m_{\alpha, K} = fg$, for $f, g \in K[x]$. In that case, $f(\alpha) g(\alpha) = 0$, meaning that either $f(\alpha) = 0$ or $g(\alpha) = 0$. Without loss of generality, assume that $f(\alpha) = 0$, meaning that $f \in \ker ev_\alpha = (m_{\alpha, K})$. Hence, $g$ is a unit in $K[x]$, so $m_{\alpha, K}$ is irreducible.
    \end{proof}

    \begin{lemma}
        Let $F|K$ be a field extension and let $\alpha \in F$ be algebraic over $K$. Then,
        \[K[x]/(m_{\alpha, K}) \cong \operatorname{Im} ev_\alpha = K(\alpha) \subseteq F.\]
    \end{lemma}
    \begin{proof}
        By the First isomorphism theorem, we know that 
        \[K[x]/(m_{\alpha, K}) \cong \operatorname{Im} ev_\alpha.\]
        Since $m_{\alpha, K}$ is irreducible, we know that $\operatorname{Im} ev_\alpha$ is a field. Since the image $\operatorname{Im} ev_\alpha$ is a field containing $K$ (constant functions) and $\alpha$ (the image of $f(x) = x$), i.e. $K(\alpha) \subseteq \operatorname{Im} ev_\alpha$. Moreover, a field containing $K$ and $\alpha$ contains expressions in $\alpha$ with coefficients in $K$, so $K(\alpha) \supseteq \operatorname{Im} ev_\alpha$.
    \end{proof}

    \begin{lemma}
        Let $K$ be a field and let $f \in K[x]$. Then, $\dim_K(K[x]/(f)) = \deg (f)$.
    \end{lemma}
    \begin{proof}
        For any $g \in K[x]$, the division algorithm tells us that $g = fq + r$, for $q, r \in K[x]$ with $\deg r < \deg f = d$. Hence, 
        \[K[x] = (f) \oplus K[x]_{\leq d},\]
        where $K[x]_{\leq d}$ is the space of polynomials of degree less than $d$. Hence, $K[x]/(f) \cong K[x]_{\leq d}$ as vector spaces, meaning that 
        \[\dim_K(K[x]/(f)) = \dim_K K[x]_{\leq d} = d = \deg (f).\]
    \end{proof}

    \begin{theorem}
        Let $F|K$ be a field extension and let $\alpha \in F$ be algebraic over $K$, with monic polynomial $m_{\alpha, K} \in K[x]$. Then, there is an isomorphism of fields and $K$-vector spaces
        \[K[x]/(m_{\alpha, K}) \to K(\alpha)\]
        given by $f + (m_{\alpha, K}) \mapsto f(\alpha)$. In particular, $[K(\alpha) : K] = \deg (m_{\alpha, K})$. In particular, $K(\alpha)|K$ is an algebraic extension.
    \end{theorem}

    \begin{definition}
        Let $F|K$ be a field extension. We say that $F|K$ is \emph{simple} if $F = K(\alpha)$ for some $\alpha \in F$.
    \end{definition}

    \begin{proposition}
        Let $F|K$ be a field extension and let $\alpha, \beta \in F$ be algebraic over $K$, with $K(\alpha)|K$ and $K(\beta)|K$ simple algebraic extensions. Then, there exists a field isomorphism $\theta \colon K(\alpha) \to K(\beta)$ fixing all elements of $K$ such that $\theta(\alpha) = \beta$ if and only if $\alpha$ and $\beta$ have the same minimal polynomial over $K$.
    \end{proposition}
    \begin{proof}
        Assume first that there exists a field isomorphism $\theta \colon K(\alpha) \to K(\beta)$. In that case, let 
        \[m_{\alpha, K} = x^n + a_{n-1} x^{n-1} + \dots + a_1 x + a_0 \in K[x].\]
        We know that $m_{\alpha, K}(\alpha) = 0$, meaning that
        \begin{align*}
            m_{\alpha, K}(\beta) &= \beta^n + a_{n-1} \beta^{n-1} + \dots + a_1 \beta + a_0 \\
            &= \theta(\alpha)^n + a_{n-1} \theta(\alpha)^{n-1} + \dots + a_1 \theta(\alpha) + a_0 \\
            &= \theta(\alpha^n + a_{n-1} \alpha^{n-1} + \dots + a_1 \alpha + a_0) \\
            &= \theta(f(\alpha)) = \theta(0) = 0.
        \end{align*}
        Since $m_{\alpha, K}$ is irreducible over $K$, we find that $m_{\alpha, K} = m_{\beta, K}$. That is, $\alpha$ and $\beta$ have the same minimal polynomial over $K$.

        Now, assume that $\alpha$ and $\beta$ have the same minimal polynomial $f$ over $K$. We know that
        \[K(\alpha) \cong K[x]/(f) \cong K(\beta).\]
        We have isomorphisms $\varphi_\alpha \colon K[x]/(f) \to K(\alpha)$ and $\varphi_\beta \colon K[x]/(f) \to K(\beta)$. Define $\theta = \varphi_\beta \circ \varphi_{\alpha}^{-1}$. Then, for all $k \in K$,
        \[\theta(k) = \varphi_\beta(\varphi_\alpha^{-1}(k)) = \varphi_\beta(k + (f)) = k.\]
        So, $\theta$ is a field isomorphism fixing $K$.
    \end{proof}

    \begin{theorem}[Kronecker's Theorem]
        Let $K$ be a field and let $f \in K[x]$ be a polynomial. Then, there exists a field extension $F|K$ and an $\alpha \in F$ such that $f(\alpha) = 0$.
    \end{theorem}
    \begin{proof}
        Without loss of generality, assume that $f$ is irreducible. In that case, $F = K[x]/(f)$ is a field, with $F|K$ is a field extension. Now, let $\alpha \in F$ by $\alpha = x + (f)$. Then,
        \[f(\alpha) = f(x) + (f) = 0.\]
        %  We know that there exist canonical maps $\varphi \colon K \to K[x]$ and $\psi \colon K[x] \to F$. So, define the composition $\iota \colon K \to F$. This is a ring homomorphism, and since $F$ is a field, this is necessarily injective. 
    \end{proof}

    \begin{definition}
        Let $\overline{K}|K$ be a field extension. We say that $\overline{K}|K$ is \emph{algebraic field extension} if $\overline{K}$ is algebraically closed.
    \end{definition}

    \begin{theorem}
        Every field has an algebraic closure.
    \end{theorem}

    \begin{theorem}
        Let $K$ be a subfield of $\mathbb{C}$. Then,
        \[\overline{K} = \{\alpha \in \mathbb{C} \mid \alpha \textrm{ algebraic over } K\}.\]
    \end{theorem}
    \begin{proof}
        Let
        \[L = \{\alpha \in \mathbb{C} \mid \alpha \textrm{ algebraic over } K\}.\]
        We first claim that $L$ is a field. So, let $\alpha_1, \alpha_2 \in L$. Then, $\alpha_1, \alpha_2$ are algebraic over $K$. In that case, we know that $[K(\alpha, \beta) : K]$ is finite, and hence algebraic. In particular, we find that $\alpha - \beta \in K(\alpha, \beta)$ is algebraic over $K$, and $\alpha \beta^{-1} \in K(\alpha, \beta)$ is algebraic over $K$ if $\beta$ is non-zero. Hence, $L$ is a field.

        By definition, $L$ is algebraic over $K$, i.e. $L \subseteq \overline{K}$. Now, we show that $\overline{K} = L$. So, let $f \in K[x]$ be a non-constant polynomial. We know that $f$ has roots in $\mathbb{C}$- $\alpha_1, \dots, \alpha_n$. Define the field $M = L(\alpha_1, \dots, \alpha_n)$. Since $M|L$ is finite, it is algebraic. Now, since $M|L$ and $L|K$ are both algebraic, $M|K$ is algebraic. In particular, $\alpha_1, \dots, \alpha_n$ are algebraic over $K$, meaning that they lie in $L$. Hence, $L$ is algebraically closed. Hence, $L = \overline{K}$.
    \end{proof}

    % TODO: 6.6 - 6.8

    \newpage

    \section{Normal and Separable Extensions}
    \begin{definition}
        An \emph{algebra} over a ring $K$ is a ring homomorphism $\eta \colon K \to F$, for some field $K$. A homomorphism between two $K$-algebras $\eta \colon K \to F$ and $\eta' \colon K \to F'$ is a ring homomorphism $f \colon F \to F'$ such that $f \circ \eta = \eta'$.
    \end{definition}

    \begin{definition}
        Let $F|K$ be a field extension. We denote by $\operatorname{Aut}(F|K)$ the set of all $K$-algebra isomorphism $F \to F$, considered as a group under composition.
    \end{definition}

    \begin{definition}
        Let $K$ be a field and let $f \in K[x]$. An extension field $F$ of $K$ is called a \emph{splitting field} for $f$ if it factorises into linear factors over $F$, and if there is no intermediate field $K \subseteq E \subsetneq F$ with this property.
    \end{definition}

    \begin{theorem}
        Let $K$ be a field. For every $f \in K[x]$, there exists a splitting field $F|K$.
    \end{theorem}
    \begin{proof}
        If $f$ factorises into linear factors over $K$, then $K$ is the splitting field of $f$. Otherwise, $f$ has an irreducible factor $g$. In that case, let $F = K[x]/(g)$. Since $g$ is irreducible, $(g)$ is a maximal ideal, and so $F$ a field. Now, let $\alpha = x + (g) \in F$. We have
        \[g(\alpha) = g(x) + (g) = 0 + (g),\]
        so $g$ has a root in $K$. Hence, $g(x) = (x - \alpha)h(x)$, where $\deg h = \deg g - 1$. We can continue this process to fully factorise $f$- it will take at most $\deg f$ steps.

        The resulting field $F$ must be the splitting field for $f$- by construction, $f$ splits into linear factors in $F$. Moreover, for an intermediate subfield $K \subseteq L \subsetneq F$, we know that $L$ does not contain one of the roots of $f$ (by construction), and so $f$ cannot split into linear factors in $L$.
    \end{proof}
    \noindent This field can be taken as the algebraic closure $\overline{K}$ of $K$.

    \begin{theorem}
        Let $\phi \colon K_1 \to K_2$ be a field isomorphism. Moreover, let $F_1|K_1$ be a splitting field for $f \in K_1[x]$ and let $F_2|K_2$ be a splitting field $\phi(f) \in K_2[x]$. Then, there exists an isomorphism $\Phi \colon F_1 \to F_2$ such that $\iota_1 \circ \Phi = \phi \circ \iota_2$, where $\iota_1 \circ K_1 \to F_1$ and $\iota_2 \colon K_2 \to F_2$ be inclusion maps.
    \end{theorem}
    \begin{proof}
        % We prove this by induction on roots of $f$ not present in $K_1$. If all roots of $f$ are present in $K_1$, then we know that all roots of $\phi(f)$ are present in $K_2$. Hence, $F_1 = K_1$ and $F_2 = K_2$. So, we set $\Phi = \phi$.

        % Otherwise, let $\alpha$ be a root of $f$ not present in $K_1$. In that case, there exists a minimum polynomial $g_\alpha \in K_1[x]$ of $\alpha$. So, $\phi(g_\alpha) \in K_2[x]$ is irreducible.
    \end{proof}

    \begin{corollary}
        Let $K$ be a field and let $f \in K[x]$. If $F_1$ and $F_2$ are splitting fields for $f$, then there exists an isomorphism $F_1 \cong F_2$ of $K$-algebras.
    \end{corollary}
    \begin{proof}
        
    \end{proof}

    \begin{definition}
        Let $F|K$ be a field extension. We say that the extension $F|K$ is \emph{normal} if for all polynomial $f \in K[x]$ irreducible with a root in $F$, $f$ splits into linear factors in $F[x]$.
    \end{definition}

    \begin{theorem}
        Let $E|K$ be a field extension. Then, $E|K$ is a finite, normal extension if and only if $E|K$ is the splitting field of some $f \in K[x]$.
    \end{theorem}
    \begin{proof}
        First, assume that $E|K$ is a finite, normal extension. In that case, $E = K(\alpha_1, \dots, \alpha_n)$, for $\alpha_1, \dots, \alpha_n \in E$. Since $\alpha_i$ are algebraic over $K$, we can find minimal polynomial $f_i \in K[x]$ of $\alpha_i$. Since $E|K$ is normal, we know that $f_i$ split over $E$. Hence, $f = f_1 \dots f_n$ splits over $E$. Since $E$ is generated by the roots of $f$, $E$ must be the splitting field of $f$.

        Now, assume that $E|K$ is the splitting field of $f \in K[x]$. In that case, $E = K(\alpha_1, \dots, \alpha_n)$, for $\alpha_1, \dots, \alpha_n \in \overline{K}$. So, $E|K$ is a finite extension. Now, let $g \in K[x]$ be an irreducible polynomial with a root in $E$. Define $F$ to be the splitting field of $fg$. Since all roots of $f$ are roots of $fg$, we find that $E \subseteq F$. Let $\beta_1, \beta_2 \in F$ be roots of $g$. We claim that
        \[[E(\beta_1) : E] = [E(\beta_2) : E].\]
        Consider the field towers
        \[K \subseteq K(\beta_1) \subseteq E(\beta_1) \subseteq F \qquad K \subseteq K(\beta_2) \subseteq E(\beta_2) \subseteq F.\]
        Hence,
        \begin{align*}
            [E(\beta_1) : K(\beta_1)] \cdot [K(\beta_1) : K] &= [E(\beta_1) : K] = [E(\beta_1) : E] \cdot [E : K] \\
            [E(\beta_2) : K(\beta_2)] \cdot [K(\beta_2) : K] &= [E(\beta_2) : K] = [E(\beta_2) : E] \cdot [E : K].
        \end{align*}
        Since $g$ is irreducible over $K$, we find that $K(\beta_1) \cong K(\beta_2)$, meaning that $[K(\beta_1) : K] = [K(\beta_2) : K]$. Moreover, since $E(\beta_j)$ is the splitting field of $f$ over $K(\beta_j)$ for $j = 1, 2$, we have $E(\beta_1) \cong E(\beta_2)$. So,
        \[[E(\beta_1) : K(\beta_1)] = [E(\beta_2) : K(\beta_2)].\]
        So,
        \begin{align*}
            [E(\beta_1) : E] &= \frac{[E(\beta_1) : K(\beta_1)] \cdot [K(\beta_1) : K]}{[E : K]} \\
            &\frac{[E(\beta_2) : K(\beta_2)] \cdot [K(\beta_2) : K]}{[E : K]} = [E(\beta_2) : K].
        \end{align*}
        We know that $\beta \in E$ if and only if $[E(\beta_1) : E] = 1$. Since $g$ has a root in $E$, all the roots of $g$ are contained in $E$. Hence, $E|K$ is normal.
    \end{proof}

    \begin{theorem}
        Let $F|K$ be a finite normal extension. If $K \subseteq E \subseteq F$ is an intermediate field, then any $K$-algebra homomorphism $\sigma \colon E \to F$ extends to a $K$-algebra homomorphism $F \to F$.
    \end{theorem}
    \begin{proof}
        Note that $\sigma$ is injective, so $E \cong \sigma(E)$. We know that $F$ is the splitting field of some polynomial $f \in K[x]$. So, $F = K(\alpha_1, \dots, \alpha_n)$, where $\alpha_i$ are roots of $f$ not in $K$. Now, let $f_1 \in E[x]$ be the minimal polynomial of $\alpha_1$. In that case, $\sigma(f_1)$ is irreducible. Moreover, $f_1$ divides $f$, meaning that $\sigma(f_1)$ divides $\sigma(f) = f$. Since $\alpha_1$ and $\beta_1$ have the same minimal polynomial, we can extend $\sigma$ to $E(\alpha_1) \to \sigma(E)(\beta_1)$. We can continue this inductively to extend the homomorphism to $F$.
    \end{proof}

    \begin{theorem}
        Let $E|K$ be a finite extension. Then, $E|K$ is normal if and only if there exists a finite normal extension $F|K$ such that $K \subseteq E \subseteq F$, and for every $K$-homomorphism $\sigma \colon E \to F$, we have $\sigma(E) = E$.
    \end{theorem}
    \begin{proof}
        If $E|K$ is normal, we can take $F = E$- for a $K$-homomorphism $\sigma \colon E \to F$, we have $\sigma(E) = F = E$. Now, assume that there exists a finite normal extension $F|K$ such that $K \subseteq E \subseteq F$, and for every $K$-homomorphism $\sigma \colon E \to F$, we have $\sigma(E) = E$. Let $f \in K[x]$ have a root $\alpha \in E$. Without loss of generality, assume that $f$ is irreducible over $K$. In that case, since $F|K$ is normal, $f$ splits over $F$. Now, let $\beta \in F$ be another root of $f$. We know that we can extend $\sigma$ to $K(\alpha) \to K(\beta) \subseteq F$. Hence, we can further extend $\sigma$ to $F \to F$. In that case, we have $\beta = \sigma(\alpha) \in E$, meaning that $E|K$ is normal.
    \end{proof}

    \begin{proposition}
        Let $K(\alpha)|K$ is finite, then 
        \[[\operatorname{Aut} (K(\alpha) | K)] \leq [K(\alpha) : K].\]
    \end{proposition}
    \begin{proof}
        Let $f \in K[x]$ be the minimal polynomial of $\alpha$. We know that $[K(\alpha) : K] = \deg f$. Now, let $\alpha_1, \dots, \alpha_m$ be the roots of $\alpha$. Since $f$ has at most $\deg f$ distinct roots, we have $m \leq \deg f$. A $K$-homomorphism $K(\alpha) \to K(\alpha)$ permutes the roots of $\alpha$. Moreover, it is determined by where it sends $\alpha$. Hence, 
        \[[\operatorname{Aut} (K(\alpha) | K)] \leq m \leq [K(\alpha) : K].\]
    \end{proof}

    \begin{definition}
        Let $K$ be a field, and $F|K$ an algebraic extension.
        \begin{enumerate}
            \item An irreducible polynomial $f \in K[x]$ is \emph{separable} if every root of $f$ is a splitting field $F$ of $f$ is simple (i.e. appears with multiplicity 1).
            \item An element $\alpha \in F$ is called \emph{separable} if its minimal polynomial is separable.
            \item $F|K$ is called \emph{separable} if every element of $F$ is separable.
        \end{enumerate}
    \end{definition}

    \begin{proposition}
        Let $E|K$ be a finite, normal, separable, simple extension. Then, $[E : K] = |\operatorname{Aut}(E|K)|$.
    \end{proposition}
    \begin{proof}
        Let $E = K(\alpha)$, for some $\alpha \in E$, and let $\alpha_1, \dots, \alpha_m$ be the roots of the minimal polynomial $f$ of $\alpha$. Without loss of generality, assume that $\alpha_1 = \alpha$. Since $E|K$ is separable, we have $m = \deg f$. We know that there exists a $K$-homomorphism $K(\alpha) \to K(\alpha_i)$ that maps $\alpha$ to $\alpha_i$, for each $1 \leq i \leq m$. Hence,
        \[[E : K] = [K(\alpha) : K] = [K(\alpha_i) : K].\]
        Since $K(\alpha_i) \subseteq E$, this implies that $K(\alpha_i) = E$. So, there are $\deg f$ distinct $K$-automorphisms of $E$, i.e. $[E : K] = |\operatorname{Aut}(E|K)|$.
    \end{proof}

    \begin{theorem}[Primitive Element Theorem]
        Let $E|K$ be a finite, separable extension. Then, there exists an $\alpha \in E$ such that $E = K(\alpha)$.
    \end{theorem}
    \begin{proof}
        First, assume that $F$ has infinitely many elements. Since $E|K$ is finite, we know that $E = K(\alpha_1, \dots, \alpha_n)$, for $\alpha_1, \dots, \alpha_n \in E$. So, inductively, we show that $K(\beta, \gamma) = K(\alpha)$ for all $\beta, \gamma \in E$. Let $f, g$ be the minimal polynomials of $\beta$ and $\gamma$ over $K$ respectively. Next, let $F$ be the splitting field of $fg \in E[x]$. Let $\beta_1 = \beta, \beta_2, \dots, \beta_m$ and $\gamma_1 = \gamma, \gamma_2, \dots, \gamma_n$ be the roots of $f$ and $g$ respectively. Since $E|K$ is separable, $\beta_1, \dots, \beta_m$ are distinct, and so are $\gamma_1, \dots, \gamma_n$. Now, since $K$ is infinite, there exists a non-zero $\alpha \in K$ such that
        \[\alpha \neq \frac{\beta_i - \beta}{\gamma_j - \gamma}\]
        for $1 \leq i \leq m$ and $2 \leq j \leq n$. Now, let $\alpha = \beta + a \gamma$. We find that $\alpha - a \gamma_j \neq \beta_i$ for $1 \leq i \leq m$ and $2 \leq j \leq n$. We know that $f$ is the minimal polynomial of $\beta$, so $f(\beta) = 0$. Now, consider the polynomial $h(x) = f(\alpha - ax) \in K(\alpha)[x]$. We find that
        \[h(\gamma) = f(\alpha - a \gamma) = f(\beta) = 0.\] 
        Moreover, $h(\gamma_j) \neq 0$ for all $2 \leq j \leq n$, meaning that $\gamma$ is the only common root of $h$ and $g$. That is, $m_{\gamma, K(\alpha)}$ divides both $h$ and $g$, we find that $m_{\gamma, K(\alpha)}$ is linear, i.e. $m_{\gamma, K(\alpha)} = x - \gamma \in K(\alpha)[x]$. So, $\gamma \in K(\alpha)$, meaning that $\beta = \alpha - a\gamma \in K(\alpha)$. Hence, $K(\alpha) = K(\beta, \gamma)$.

        Instead, if $K$ is finite, then $E$ is also finite. Since $E$ is a field, this implies that $E^*$ is cyclic, so let $E^* = \langle \alpha \rangle$. Then, $E = K(\alpha)$.
    \end{proof}

    \begin{definition}
        Let $E|K$ be a finite, separable extension, and let $\alpha \in E$ such that $E = K(\alpha)$. We say that $\alpha$ is a \emph{primitive element}.
    \end{definition}

    \begin{corollary}
        Let $E|K$ be any finite, normal, separable extension. Then,
        \[[E : K] = [Aut(E|K)].\]
    \end{corollary}
    \begin{proof}
        Since $E|K$ is finite and separable, we have $E = K(\alpha)$. Hence, $E$ is simple, meaning that 
        \[[E : K] = [Aut(E|K)].\]
    \end{proof}

    \begin{proposition}
        Let $F|E$ and $E|K$ be finite extensions. If $F|K$ is separable, then $F|E$ and $E|K$ are separable.
    \end{proposition}
    \begin{proof}
        Let $\alpha \in E$. Since $F|K$ is separable, we know that the minimal polynomial of $\alpha \in F$ is separable. Hence, $E|K$ is separable. Now, let $\alpha \in F$, and let $f \in E[x]$ be the minimal polynomial of $\alpha$, and $g \in K[x]$ the minimal polynomial of $\alpha$. Since $F|K$ is separable, $g$ has distinct roots. We have $f \mid g$, so $f$ also has distinct roots, meaning that $f$ is separable. Hence, $F|E$ is separable.
    \end{proof}

    \begin{definition}
        Let $K$ be a field. We say that $K$ is \emph{perfect} if every finite extension of $K$ is separable.
    \end{definition}

    \begin{lemma}
        Let $K$ be a field of characteristic zero, and let $f \in K[x]$ be non-zero. Let $\overline{K}$ be the algebraic closure of $K$. Then, $f$ has multiple roots in $\overline{K}$ if and only if $f$ and the derivative $f'$ have a common factor of positive degree in $K[x]$.
    \end{lemma}
    \begin{proof}
        
    \end{proof}

    \begin{proposition}
        Let $K$ be a field of characteristic zero. Then, every irreducible polynomial $f \in K[x]$ is separable. Hence, $K$ is a perfect field.
    \end{proposition}
    \begin{proof}
        
    \end{proof}

    \begin{proposition}
        Let $K$ be a finite field. Then, $K$ is perfect.
    \end{proposition}
    \begin{proof}
        
    \end{proof}
    \newpage

    \section{Galois Extensions}
    \begin{definition}
        Let $F|K$ is a field extension. We say that $F|K$ is \emph{Galois} if it is finite, normal and separable.
    \end{definition}

    \begin{corollary}
        Let $K \subseteq F \subseteq \mathbb{C}$ be fields. Then, the following are equivalent:
        \begin{enumerate}
            \item $F|K$ is Galois.
            \item $F|K$ is finite and normal.
            \item $F$ is the splitting field of some $f \in K[x]$.
        \end{enumerate}
    \end{corollary}
    \begin{proof}
        
    \end{proof}

    \begin{definition}
        Let $F|K$ be a Galois extension. Then, the automorphism group $\operatorname{Aut}(F|K)$ is the \emph{Galois group} of $F|K$.
    \end{definition}

    \begin{proposition}
        Let $F|K$ be a Galois extension and let $K \subseteq E \subseteq F$ be an intermediate field. Then, $F|E$ is a Galois extension and $\operatorname{Fix}(F, \operatorname{Aut}(F|E)) = E$.
    \end{proposition}
    \begin{proof}
        
    \end{proof}

    \begin{theorem}[The Main Theorem of Galois Theory]
        Let $F|K$ be a Galois extension, with Galois group $G = \operatorname{Aut}(F|K)$.
        \begin{enumerate}
            \item Let 
            \[M = \{E \mid K \subseteq E \subseteq F \textrm{ intermediate field}\}, \qquad N = \{H \mid H \subseteq G \textrm{ subgroup}\}.\]
            Then, there is a map $\alpha \colon M \to N$ defined by $\alpha(E) = \operatorname{AUt}(F|E)$, with inverse $\phi \colon N \to M$ given by $\phi(H) = \operatorname{Fix}(F, H)$, which are inverse bijections.

            \item The maps $\alpha$ and $\phi$ are order reversing, i.e.
            \[E_1 \subseteq E_2 \iff \alpha(E_1) \supseteq \alpha(E_2) \iff \operatorname{Aut}(F|E_2) \subseteq \operatorname{Aut}(F|E_1),\]
            and
            \[H_1 \subseteq H_2 \iff \phi(H_1) \supseteq \phi(H_2) \iff \operatorname{Fix}(F, H_2) \subseteq \operatorname{Fix}(F, H_1).\]

            \item If $K \subseteq E \subseteq F$ is an intermediate field, then $F|E$ is Galois, with
            \[[F : E] = |\operatorname{Aut}(F|E)| \textrm{ and } [E : K] = \frac{|G|}{|\operatorname{Aut}(F|E|)}.\]

            \item A subgroup $H \subseteq G$ is normal if and only if the corresponding field extension $\phi(H)|K$ is normal. In this case,
            \[\operatorname{Aut}(\varphi(H)|K) \cong G/H.\]
            Alternatively, an intermediate field extension $E|K$ is normal if and only if $\operatorname{Aut}(F|E) \vartriangleleft \operatorname{Aut}(F|K)$, in which case
            \[\operatorname{Aut}(E|K) \cong \operatorname{Aut}(F|K)/\operatorname{Aut}(F|E).\]
        \end{enumerate}
    \end{theorem}
    \begin{proof}
        
    \end{proof}
    
    % TODO: Applications
    \newpage

    \section{Solving the Quintic Equation}
    \begin{definition}
        Let $G$ be a group. We say that $G$ is \emph{solvable} if there is a composition series
        \[\{e\} = G_0 \vartriangleleft G_1 \vartriangleleft G_2 \vartriangleleft \dots \vartriangleleft G_n = G\]
        such that for every $0 \leq j < n$,
        \begin{itemize}
            \item the group $G_j$ is normal in $G_{j+1}$ and
            \item the quotient group $G_{j+1}/G_j$ is abelian.
        \end{itemize}
    \end{definition}

    \begin{lemma}
        Let $G$ be a group and $N \vartriangleleft G$. Then, $G$ is solvable if and only if $N$ and $G/N$ are solvable.
    \end{lemma}
    \begin{proof}
        
    \end{proof}

    \begin{theorem}
        Let $n \in \mathbb{Z}_{\geq 5}$. Then, $A_n$ is not simple. In particular, $S_n$ is not solvable.
    \end{theorem}
    \begin{proof}
        
    \end{proof}

    \begin{proposition}
        Let $G \leq S_5$ such that $G$ has a transposition and a 5-cycle. Then, $G = S_5$.
    \end{proposition}
    \begin{proof}
        
    \end{proof}

    \begin{definition}
        Let $F|K$ be a field extension and let $\alpha \in F$. We say that $\alpha \in F$ is a \emph{radical over $K$} if there exists an $n \in \mathbb{Z}_{\geq 1}$ such that $\alpha^n \in K$.
    \end{definition}

    \begin{definition}
        Let $F|K$ be a field extension. We say that $F|K$ is a \emph{radical extension} if
        \[F = K(\alpha_1, \dots, \alpha_m)\]
        such that $\alpha_1$ is a radical over $K$, and $\alpha_j$ is a radical over $K(\alpha_1, \dots, \alpha_{j-1})$ for $2 \leq j \leq m$. The elements $\alpha_j$ are said to form a \emph{radical sequence}.
    \end{definition}

    \begin{definition}
        Let $K \subseteq \mathbb{C}$ be a field and $f \in K[x]$. If $K \subseteq E \subseteq \mathbb{C}$ is a splitting field for $f$, then we say that $f$ is \emph{solvable by radicals} if there exists $K \subseteq E \subseteq F \subseteq \mathbb{C}$ such that $F|K$ is a radical extension.
    \end{definition}

    \begin{lemma}
        Let $K \subseteq E \subseteq \mathbb{C}$, where $E|K$ is the splitting field of $x^n - 1 \in K[x]$ and $n \in \mathbb{Z}_{\geq 1}$. Then, $\operatorname{Aut}(E|K)$ is abelian.
    \end{lemma}
    \begin{proof}
        
    \end{proof}

    \begin{lemma}
        Let $n \in \mathbb{Z}_{\geq 1}$ and let $E \subseteq \mathbb{C}$ be a subfield in which $x^n - 1$ splits. Moreover, let $a \in E$ and $F \subseteq \mathbb{C}$ be the splitting field for $x^n - a \in E[x]$. Then, $\operatorname{Aut}(F|E)$ is abelian.
    \end{lemma}
    \begin{proof}
        
    \end{proof}

    \begin{proposition}
        Let $K \subseteq \mathbb{C}$ and $a \in K$. If $F$ is the splitting field of $f(x) = x^n - a \in K[x]$, then $\operatorname{Aut}(F|K)$ is solvable.
    \end{proposition}
    \begin{proof}
        
    \end{proof}

    \begin{theorem}
        Let $K \subseteq E \subseteq F \subseteq \mathbb{C}$ be fields such that $E|K$ is normal and $F|K$ a radical extension. Then, the group $\operatorname{Aut}(E|K)$ is solvable.
    \end{theorem}
    \begin{proof}
        
    \end{proof}

    \begin{corollary}
        Let $K \subseteq \mathbb{C}$ be a field, $f \in K[x]$ and let $E|K$ be a splitting field of $f$. If $f$ is solvable by radicals, then $\operatorname{Aut}(E|K)$ is solvable.
    \end{corollary}
    \begin{proof}
        
    \end{proof}

    \begin{definition}
        Let $F|K$ be a field extension. We say that $\operatorname{Aut}(F|K)$ is the \emph{Galois group} of $f \in K[x]$ over $K$ if $F|K$ is the splitting field of $f$.
    \end{definition}

    \begin{lemma}
        Let $f \in \mathbb{Q}[x]$ be an irreducible polynomial of degree 5. If $f$ has precisely three real roots in $\mathbb{C}$, then the Galois group of $f$ over $\mathbb{Q}$ is isomorphic to the symmetric group $S_5$.
    \end{lemma}
    \begin{proof}
        
    \end{proof}

    \begin{example}
        We show that the polynomial $f(x) = x^5 - 6x + 3 \in \mathbb{Q}[x]$ is not solvable by radicals.
        % TODO
    \end{example}

\end{document}