\documentclass[a4paper, openany]{memoir}

\usepackage[utf8]{inputenc}
\usepackage[T1]{fontenc} 
\usepackage[english]{babel}

\usepackage{fancyhdr}
\usepackage{float}
\usepackage{bm}

\usepackage{amsmath}
\usepackage{amsthm}
\usepackage{amssymb}
\usepackage{enumitem}
\usepackage{multicol}
\usepackage[bookmarksopen=true,bookmarksopenlevel=2]{hyperref}
\usepackage{tikz}
\usepackage{indentfirst}

\pagestyle{fancy}
\fancyhf{}
\fancyhead[LE]{\leftmark}
\fancyhead[RO]{\rightmark}
\fancyhead[RE, LO]{Topics In Algebra}
\fancyfoot[LE, RO]{\thepage}
\fancyfoot[RE, LO]{Pete Gautam}

\renewcommand{\headrulewidth}{1.5pt}

\theoremstyle{definition}
\newtheorem{definition}{Definition}[section]
\newtheorem{example}[definition]{Example}

\theoremstyle{plain}
\newtheorem{theorem}[definition]{Theorem}
\newtheorem{lemma}[definition]{Lemma}
\newtheorem{proposition}[definition]{Proposition}
\newtheorem{corollary}[definition]{Corollary}


\chapterstyle{thatcher}
\setcounter{chapter}{3}

\begin{document}
    \chapter{Free Groups}
    \section{Introduction to Free Groups}
    \begin{definition}
        Let $S$ be a set, and fix a set $S^-$ disjoint to $S$ with a bijection $f\colon S \to S^-$, and a singleton set $\{e\}$. Denote $X_S = S \cup S^- \cup \{1\}$. We define the \emph{inverse map} $-1\colon X_S \to X_S$ by
        \[s^{-1} = \begin{cases}
            e & s = e \\
            \varphi(s) & s \in S \\
            \varphi^{-1}(s) & s \in S^-.
        \end{cases}\]
    \end{definition}

    \begin{definition}
        Let $S$ be a set. A \emph{word} on $S$ is an infinite tuple $(s_1, s_2, \dots)$ with values in $X_S$ such that there exists an $N \in \mathbb{Z}_{\geq 1}$ such that for all $n \in \mathbb{Z}_{\geq 1}$, if $n \geq N$, then $s_n = e$. A \emph{reduced word} on $S$ is a word $(s_1, s_2, \dots)$ such that:
        \begin{itemize}
            \item if $s_N = e$ for some $N \geq 1$, then $s_n = e$ for all $n \geq N$;
            \item if $s_i \neq e$, then $s_{i+1} \neq s_1^{-1}$ for all $n \in \mathbb{Z}_{\geq 1}$.
        \end{itemize}
        We denote a reduced word $(s_1, s_2, \dots, s_n, e, e, \dots)$ by $s_1s_2 \dots s_n$, where $s_n \neq e$. The set of all reduced words is denoted by $F(S)$. We have the inclusion map $\iota \colon S \to F(S)$ given by $\iota(s) = (s, e, e, \dots)$. We also denote $e = (e, e, e, \dots)$, and call it \emph{identity element}.
    \end{definition}

    \begin{definition}
        Let $S$ be a set. Define the operation $\cdot \colon F(S) \to F(S)$ by 
        \[s_1 \dots s_n \cdot t_1 \dots t_k = \]
        The operation is called \emph{concatenation}.
    \end{definition}

    \begin{proposition}
        Let $S$ be a set. Then, $F(S)$ is a group under concatenation.
    \end{proposition}
    \begin{proof}
        
    \end{proof}

    \begin{proposition}
        Let $S$ be a set, $G$ be a group, and $f: S \to G$ be a map. Then, there exists a unique homomorphism $\varphi: F(S) \to G$ such that $\varphi(s) = f(s)$ for all $s \in S$.
    \end{proposition}
    \begin{proof}
        
    \end{proof}

    \begin{corollary}
        Let $S$ be a set. Then, $F(S)$ is unique.
    \end{corollary}
    \begin{proof}
        
    \end{proof}
    \newpage

    \section{Group Relation and Presentation}
\end{document}